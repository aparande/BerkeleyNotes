\documentclass{article}
\usepackage{amssymb}
\usepackage{amsmath}
\usepackage{mathtools}
\usepackage{cancel}
\usepackage{tikz}
\usepackage{hyperref}
\usepackage{circuitikz}
\usepackage{float}
\usepackage{afterpage}
\usetikzlibrary{calc}
\newtheorem{theorem}{Theorem}
\newtheorem{definition}{Definition}
\newtheorem{corollary}{Corollary}
\newtheorem{proof}{Proof}

\DeclareMathOperator*{\argmin}{argmin}

\begin{document}
\title{CS70 Course Notes}
\author{Anmol Parande}
\date{Fall 2019 - Professors Alistair Sinclair and Yun Song}
\maketitle
\textbf{Disclaimer: }These notes reflect CS60 when I took the course (Fall 2019). They may not accurately reflect current course content, so use at your own risk.
If you find any typos, errors, etc, please report them on the \href{https://github.com/parandea17/BerkeleyNotes}{GitHub repository}.\\
\tableofcontents
\newpage
\section{Propositional Logic}
\begin{definition}
    A proposition $P$ is a statement that is either true or false
\end{definition}
Propositions can depend on one or more variables. This is denoted by $P(x, y, ...)$
\subsection{Connectives}
\begin{definition}
    A connective is an operator which joins two or more propositions together in some way
\end{definition}
Connectives are fully defined by a \textbf{Truth Table} which enumerates the values of the connective
given all possible combination of inputs.
\begin{definition}
    The base connectives are those which can be combined to create any other connective.
    \begin{itemize}
        \item $\land$: P AND Q
        \item $\lor$: P OR Q
        \item $\lnot$: NOT P
    \end{itemize}
\end{definition}
One important connective is the \textbf{implies} connective.

\begin{center}
    \begin{tabular}{c|c|c} 
     P & Q & $P \implies Q$ \\
     \hline
     T & T & T \\ 
     T & F & F \\
     F & T & T \\
     F & F & T \\ 

    \end{tabular}
\end{center}
Notice that $P \implies Q$ has the same truth table as $\lnot P \lor Q$. This means they are equivalent.
\begin{definition}
    The contrapositive of $P \implies Q$ is $\lnot Q \implies \lnot P$
\end{definition}
\textbf{Notice: }The contrapositive is logically equivalent to the original statement
\begin{definition}
    The converse of $P \implies Q$ is $Q \implies P$
\end{definition}
\textbf{Notice: }This is not always equivalent to the original statement
The \textbf{if and only if} connective $P \iff Q$ is equivalent to $(P \implies Q) \land (Q \implies P)$
\begin{center}
    \begin{tabular}{c|c|c} 
     P & Q & $P \iff Q$ \\
     \hline
     T & T & T \\ 
     T & F & F \\
     F & T & F \\
     F & F & T \\ 
    \end{tabular}
\end{center}
\subsection{Quantifiers}
Quantifiers help introduce variables into our propositions.
\begin{itemize}
    \item $\forall$: For all
    \item $\exists$: There exists
\end{itemize}
\textbf{Important: }The order of quantifiers matters.
\subsection{DeMorgan's Laws}
\begin{itemize}
    \item $\lnot(P \land Q) \equiv \lnot P \lor \lnot Q$
    \item $\lnot(P \lor Q) \equiv \lnot P \land \lnot Q$
    \item $\lnot(\forall x P(x)) \equiv \exists x (\lnot P(x))$
    \item $\lnot(\exists x P(x)) \equiv \forall x (\lnot P(x))$
\end{itemize}
\end{document}