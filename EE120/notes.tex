\documentclass{article}
\usepackage{amssymb}
\usepackage{amsmath}
\usepackage{mathtools}
\usepackage{cancel}
\usepackage{tikz}
\usepackage{hyperref}
\usepackage{circuitikz}
\usepackage{float}
\usepackage{afterpage}
\usetikzlibrary{calc}
\newtheorem{theorem}{Theorem}
\newtheorem{definition}{Definition}
\newtheorem{corollary}{Corollary}
\newtheorem{proof}{Proof}

\DeclareMathOperator*{\argmin}{argmin}

\begin{document}
\title{EE120 Course Notes}
\author{Anmol Parande}
\date{Fall 2019 - Professor Murat Arcak}
\maketitle
\textbf{Disclaimer: }These notes reflect 120 when I took the course (Fall 2019). They may not accurately reflect current course content, so use at your own risk.
If you find any typos, errors, etc, please raise an issue on the \href{https://github.com/parandea17/BerkeleyNotes}{GitHub repository}.\\
\tableofcontents
\newpage
\section{Introduction to Signals and Systems}
\subsection{Types of Signals}
\begin{definition}
    A signal is a function of one or more variables
\end{definition}
\begin{definition}
    A signal $x(t)$ is continuous if $x: \mathbb{R} \rightarrow \mathbb{R}$
\end{definition}
\begin{definition}
    A signal $x[n]$ is discrete if $x: \mathbb{Z} \rightarrow \mathbb{R}$
\end{definition}
\begin{definition}
    The unit impulse is defined as 
    \[
        \delta[n] = \left\{
            \begin{array}{cc}
                1, \text{if } n=0\\
                0, \text{ else}
            \end{array}
            \right\}
    \]
\end{definition}
\begin{definition}
    The unit step is defined as 
    \[
        u[n] = \left\{
            \begin{array}{cc}
                1, \text{if } n \geq 0\\
                0, \text{ else}
            \end{array}
            \right\}
    \]
\end{definition}
\subsection{Signal transformations}
Signals can be transformed by modifying the variable.
\begin{itemize}
    \item $x(t - \tau)$: Shift a signal left by $\tau$ steps.
    \item $x(-t)$: Rotate a signal about the $t=0$
    \item $x(kt)$: Stretch a signal by a factor of $k$
\end{itemize}
These operations can be combined to give more complex transformations.
For example, $y(t) = x(\tau - t) = x(-(t-\tau))$ flips $x$ and shifts it right by $\tau$ timesteps.
This is equivalent to shifting $x$ left by $\tau$ timesteps and then flipping it.
\subsection{Systems and their properties}
\begin{definition}
    A system is a process by which input signals are transformed to output signals
\end{definition}
\begin{definition}
    A memoryless system has output which is only determined by the input's present value
\end{definition}
\begin{definition}
    A causal system has output which only depends on input at present or past times
\end{definition}
\begin{definition}
    A stable system produces bounded output when given a bounded input. By extension,
    this means an unstable system is when $\exists$ a bounded input that makes the output unbounded.
\end{definition}
\begin{definition}
    A system is time-invariant if the original input $x(t)$ is transformed to $y(t)$, then
    $x(t-\tau)$ is transformed to $y(t-\tau)$
\end{definition}
\begin{definition}
    A system $f(x)$ is linear if and only if
    \begin{itemize}
        \item If $y(t) = f(x(t))$, then $f(a x(t)) = a y(t)$ (Scaling)
        \item If $y_1(t) = f(x_1(t))$ and $y_2(t) = f(x_2(t))$, then $f(x_1(t) + x_2(t)) = y_1(t) + y_2(t)$ (Superposition)
    \end{itemize}
\end{definition}
\textbf{Notice: } The above conditions on linearity require that $x(0) = 0$ because if $a = 0$, then we need $y(0) = 0$ for scaling to be satisfied
\begin{definition}
    The impulse response of a system $f[x]$ is $h[n] = f[\delta[n]]$, which is how it response to an impulse input. 
\end{definition}
\end{document}