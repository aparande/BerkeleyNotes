\documentclass{article}
\usepackage{amssymb}
\usepackage{amsmath}
\usepackage{mathtools}
\usepackage{cancel}
\usepackage{tikz}
\usepackage{hyperref}
\usepackage{circuitikz}
\usepackage{float}
\usepackage{afterpage}
\usetikzlibrary{calc}
\newtheorem{theorem}{Theorem}
\newtheorem{definition}{Definition}
\newtheorem{corollary}{Corollary}
\newtheorem{proof}{Proof}

\DeclareMathOperator*{\argmin}{argmin}

\begin{document}
\title{EE120 Course Notes}
\author{Anmol Parande}
\date{Fall 2019 - Professor Murat Arcak}
\maketitle
\textbf{Disclaimer: }These notes reflect 120 when I took the course (Fall 2019). They may not accurately reflect current course content, so use at your own risk.
If you find any typos, errors, etc, please raise an issue on the \href{https://github.com/parandea17/BerkeleyNotes}{GitHub repository}.\\
\tableofcontents
\newpage
\section{Introduction to Signals and Systems}
\subsection{Types of Signals}
\begin{definition}
    A signal is a function of one or more variables
\end{definition}
\begin{definition}
    A signal $x(t)$ is continuous if $x: \mathbb{R} \rightarrow \mathbb{R}$
\end{definition}
\begin{definition}
    A signal $x[n]$ is discrete if $x: \mathbb{Z} \rightarrow \mathbb{R}$
\end{definition}
\subsubsection{Properties of the Unit Impulse}
\begin{definition}
    The unit impulse in discrete time is defined as 
    \[
        \delta[n] = \left\{
            \begin{array}{cc}
                1, \text{if } n=0\\
                0, \text{ else}
            \end{array}
            \right\}
    \]
\end{definition}
\begin{itemize}
    \item $f[n]\delta[n] = f[0]\delta[n]$
    \item $f[t]\delta[n-N] = f[N]\delta[n-N]$
\end{itemize}
\begin{definition}
    The unit impulse in continuous time is the dirac delta function
    $$\delta(t)=lim_{\Delta\rightarrow 0}\delta_{\Delta}(t)$$
    \[
        \delta_{\Delta}=\left\{
            \begin{array}{c}
                \frac{1}{\Delta},\text{ if }\ge 0 \\
                0 else
            \end{array}
        \right\}
    \]
\end{definition}
\begin{itemize}
    \item $f(t)\delta(t) = f(0)\delta(t)$
    \item $f(t)\delta(t-\tau) = f(\tau)\delta(t-\tau)$
    \item $\delta(at) = \frac{1}{|a|}\delta(t)$
\end{itemize}
\begin{definition}
    The unit step is defined as 
    \[
        u[n] = \left\{
            \begin{array}{cc}
                1, \text{if } n \geq 0\\
                0, \text{ else}
            \end{array}
            \right\}
    \]
\end{definition}
\subsection{Signal transformations}
Signals can be transformed by modifying the variable.
\begin{itemize}
    \item $x(t - \tau)$: Shift a signal left by $\tau$ steps.
    \item $x(-t)$: Rotate a signal about the $t=0$
    \item $x(kt)$: Stretch a signal by a factor of $k$
\end{itemize}
These operations can be combined to give more complex transformations.
For example, $y(t) = x(\tau - t) = x(-(t-\tau))$ flips $x$ and shifts it right by $\tau$ timesteps.
This is equivalent to shifting $x$ left by $\tau$ timesteps and then flipping it.
\subsection{Convolution}
\begin{definition}
    The convolution of two signals in discrete time
    $$(x*h)[n] = \sum_{k=-\infty}^{\infty}{x[k]h[n-k]}$$
\end{definition}
\begin{definition}
    The convolution of two signals in continuous time
    $$(x*h)(t) = \int_{\infty}^{\infty}{x(\tau)h(t-\tau)d\tau}$$
\end{definition}
While written in discrete time, these properties apply in continuous time as well.
\begin{itemize}
    \item $(x*\delta)[n] = x[n]$
    \item $x[n]*\delta[n-N]=x[n-N]$
    \item $(x*h)[n] = (h*x)[n]$
    \item $x * (h_1 + h_2) = x*h_1 + x*h_2$
    \item $x * (h_1 * h_2) = (x * h_1) * h_2$
\end{itemize}
\subsection{Systems and their properties}
\begin{definition}
    A system is a process by which input signals are transformed to output signals
\end{definition}
\begin{definition}
    A memoryless system has output which is only determined by the input's present value
\end{definition}
\begin{definition}
    A causal system has output which only depends on input at present or past times
\end{definition}
\begin{definition}
    A stable system produces bounded output when given a bounded input. By extension,
    this means an unstable system is when $\exists$ a bounded input that makes the output unbounded.
\end{definition}
\begin{definition}
    A system is time-invariant if the original input $x(t)$ is transformed to $y(t)$, then
    $x(t-\tau)$ is transformed to $y(t-\tau)$
\end{definition}
\begin{definition}
    A system $f(x)$ is linear if and only if
    \begin{itemize}
        \item If $y(t) = f(x(t))$, then $f(a x(t)) = a y(t)$ (Scaling)
        \item If $y_1(t) = f(x_1(t))$ and $y_2(t) = f(x_2(t))$, then $f(x_1(t) + x_2(t)) = y_1(t) + y_2(t)$ (Superposition)
    \end{itemize}
\end{definition}
\textbf{Notice: } The above conditions on linearity require that $x(0) = 0$ because if $a = 0$, then we need $y(0) = 0$ for scaling to be satisfied
\begin{definition}
    The impulse response of a system $f[x]$ is $h[n] = f[\delta[n]]$, which is how it response to an impulse input. 
\end{definition}
\subsection{Exponential Signals}
Exponential signals are important because they can succinctly represent
complicated signals using complex numbers. This makes analyzing them much easier.
$$x(t) = e^{st}, x[n] = z^n (s, z \in \mathbb{C})$$
\begin{definition}
    The frequency response of a system is how a system responds to a purely oscillatory signal
\end{definition}
\section{Linear Time-Invariant Systems}
\begin{definition}
    LTI systems are ones which are both linear and time-invariant.
\end{definition}
\subsection{Impulse Response of LTI systems}
LTI systems are special systems because their output can be determined entirely the impulse response $h[n]$.
\subsubsection{The discrete case}
We can think of the original signal $x[n]$ in terms of the impulse function.
$$x[n] = x[0]\delta[n]+x[1]\delta[n-1]+... = \sum_{k=-\infty}^{\infty}{x[k]\delta[n-k]}$$
This signal will be transformed in some way to get the output $y[n]$.
Since the LTI system applies a functional $F$ and the LTI is linear and time-invariant,
$$y[n] = F(\sum_{k=-\infty}^{\infty}{x[k]\delta[n-k]}) = \sum_{k=-\infty}^{\infty}{x[k]F(\delta[n-k])} = \sum_{k=-\infty}^{\infty}{x[k]h[n-k]}$$
Notice this operation is the convolution between the input and the impulse response.
$$y(t) = \int_{\infty}^{\infty}{x(\tau)h(t-\tau)}$$
\subsubsection{The continuous case}
We can approximate the function by breaking it into intervals of length $\Delta$.
$$x(t) \approx \sum_{k=-\infty}^{\infty}{x(k\Delta)\delta_{\Delta}(t-k\Delta)\Delta}$$
$$x(t) = lim_{\Delta \rightarrow 0}\sum_{k=-\infty}^{\infty}{x(k\Delta)\delta_{\Delta}(t-k\Delta)\Delta}$$
After applying the LTI system to it,
$$y(n) = \int_{-\infty}^{\infty}{x(\tau)h(t-\tau)}$$
Notice this operation is the convolution between the input and the impulse response.
\subsection{Determining Properties of an LTI system}
Because an LTI system is determined entirely by its impulse response, we can determine its properties from the impulse response.
\subsubsection{Causality}
\begin{theorem}
    An LTI system is causal when $h[n] = 0, \forall n < 0$
\end{theorem}
\begin{proof}
Assume $h[n] = 0, \forall n < 0$
$$y[n] = (x*h)[n] = \sum_{k=-\infty}^{\infty}{x[n-k]h[k]}=\sum_{k=0}^{\infty}{x[n-k]h[k]}$$
\end{proof}
Notice that this does not depend on time steps prior to $n=0$
\subsubsection{Memory}
\begin{theorem}
    An LTI system is memoryless if $h[n]=0, if \forall n \ne 0$
\end{theorem}
\subsubsection{Stability}
\begin{theorem}
    A system is stable if $\sum_{n=-\infty}^{\infty}{|h[n]|}$ converges.
\end{theorem}
\begin{proof}
    \textbf{\\1. } Assume $|x[n]| \le B_x$ to show $|y[n]| < D$ where D is some bound.
    $$|y[n]| = |\sum_{k=-\infty}^{\infty}{x[n-k]h[k]}| \le \sum_{k}{|x[n-k]h[k]|} = \sum_{k}{|x[n-k]||h[k]|}\le B_x\sum_{k}{|h[k]}$$
    This means as long as $\sum_{k}{|h[k]}$ converges, $y[n]$ will be bounded.\\
    \textbf{\\2. } Assume $\sum_{n}{|h[n]|}$ does not converge. Show that the system is unstable.
    Choose $x[n]=sgn\{h[-n]\}$
    $$y[n]=\sum_{k}{x[n-k]h[k]}$$ so 
    $$y[0] = \sum_{k}{x[-k]h[k]} = \sum_{k}{|h[k]|}$$
    And this is unbounded, so $y[n]$ is unbounded.
\end{proof}
\subsection{Frequency Response}
If we pass a complex exponential into an LTI system, the output signal is the same signal but scaled.
In otherwise, it is an eigenfunction of LTI systems.
$$y(t)=\int_{-\infty}^{\infty}{e^{s(t-\tau)}h(\tau)d\tau}=e^{st}\int_{-\infty}^{\infty}{e^{-s\tau}h(\tau)}$$
The integral is a constant, and the original function is unchanged.
The same analysis can be done in the discrete case.
$$y[n]=\sum_{k=-\infty}^{\infty}z^{n-k}h[k] = z^n \sum_{k=-\infty}^{\infty}z^{-k}h[k]$$
\begin{definition}
    The frequency response of a system is the output when passed a purely oscillatory signal
\end{definition}
\begin{definition}
    The transfer function of an LTI system $H(\omega)$ is how the system scales a pure tone of frequency $\omega$
    $$H(\omega):=\int_{-\infty}^{\infty}{h(\tau)e^{-j\omega\tau}d\tau}, H(\omega):= \sum_{k=-\infty}^{\infty}{h[k]e^{-j\omega k}}$$
\end{definition}

\end{document}