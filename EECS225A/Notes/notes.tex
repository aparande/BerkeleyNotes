\documentclass[12pt]{article}

\setlength{\topmargin} {-1in}

% Style modifications
\oddsidemargin  0.25in
\evensidemargin 0.25in
\textwidth      6.0in
%\textheight     8.0in
%\textheight     8.5in % Changed on 13 September 2012
\textheight     9in % Changed on 3 October 2013
\parskip        0.1in
\parindent      0.0in
\headheight     1.0in

\headsep        .25in

\usepackage{graphicx}
\usepackage{amsmath,amsfonts,amssymb,amscd,verbatim,graphicx,fancyhdr,tikz}
\usepackage{mathtools} %for minus plus sign alignment in matrices and vectors, added 13 Feb 2017

\usepackage{pgfplots}
\usetikzlibrary{arrows,automata}
\usetikzlibrary{positioning}
\usetikzlibrary{shapes.geometric}

\usepgfplotslibrary{groupplots}
\pgfplotsset{compat=1.16}

\usepackage{tikz} 
\usetikzlibrary{arrows.meta}
\usepackage{circuitikz}

\usepackage{esdiff}
\usepackage{siunitx}

\usepackage{palatino}
\usepackage{enumerate}
\usepackage{multicol}
\usepackage{listings}
\usepackage{color}
\usepackage{float}
\usepackage{esdiff} % Derivatives
\usepackage{hyperref}
\usepackage{cleveref}
\usepackage{bookmark}
\usepackage{slashbox}
\usepackage{tabularx}
\usepackage{booktabs}
\usepackage{siunitx}

\newcolumntype{Y}{>{\centering\arraybackslash}X}

\hypersetup{
    colorlinks=true,
    linkcolor=blue,
    filecolor=magenta,      
    urlcolor=blue,
}

\newcommand{\bblue}{\color[rgb]{0.2,0.2,0.7}}

\usepackage{mcode}
%Alternative: http://perso.telecom-bretagne.eu/gillesbertrand/tutorials/#sujet44
%This latter one uses the xypackage
%\usepackage{x}

\newcommand\eqnnumber{\addtocounter{equation}{1}\tag{\theequation}} % Allows us to tag the last equation in an align

\usepackage{accents}
\newcommand*{\dt}[1]{%
\accentset{\mbox{\large\bfseries .}}{#1}}
\newcommand*{\ddt}[1]{%
\accentset{\mbox{\large\bfseries .\hspace{-0.25ex}.}}{#1}}
\newcommand{\B}[1]{\mathbf{#1}}
\DeclareMathAlphabet{\V}{OML}{cmm}{b}{it}


% To get decent pdf files:
\usepackage{caption}
\usepackage{subcaption}

\newcommand{\bs}[1]{\boldsymbol{#1}}
\newcommand{\underscore}{\char`\_}
\newcommand{\tildetext}{\char`\~}
\newcommand{\defn}{\stackrel{\triangle}{=}}
\newcommand{\deriv}[2]{\frac{d{#1}}{d{#2}}}
\newcommand{\cas}{\text{cas }}

\lstset{columns=fullflexible,
    xleftmargin=0pt,
    frame=single,
    keepspaces=false,
    tabsize=2,
    breaklines=true,%
    morekeywords={matlab2tikz},
    keywordstyle=\color{blue},%
    morekeywords=[2]{1}, keywordstyle=[2]{\color{black}},
    identifierstyle=\color{black},%
    showstringspaces=false,%without this there will be a symbol in the places where there is a space
    numbers=left,%
    numberstyle={\tiny \color{black}},% size of the numbers
    numbersep=5pt, % this defines how far the numbers are from the text
    emph=[1]{for,end,break},emphstyle=[1]\color{red}, %some words to emphasise
    %emph=[2]{word1,word2}, emphstyle=[2]{style},
}

\newtheorem{theorem}{Theorem}
\newtheorem{definition}{Definition}
\newtheorem{corollary}{Corollary}
\newtheorem{proof}{Proof}

\DeclareMathOperator*{\argmin}{argmin}
\DeclareMathOperator*{\argmax}{argmax}

\newcommand{\prob}[1]{\mathbb{P}\left[#1\right]}
\newcommand{\expect}[1]{\mathbb{E}\left[#1\right]}
\newcommand{\var}[1]{Var\left(#1\right)}
\newcommand{\cov}[1]{Cov\left(#1\right)}
\newcommand{\normal}[3]{#1\sim\mathcal{N}\left(#2, #3\right)}
\newcommand{\norm}[1]{\|#1\|}
\newcommand{\vecnorm}[1]{\|\vec{#1}\|}
\newcommand{\iid}{independently and identically distributed }
\renewcommand{\arraystretch}{1.5}

\DeclareSIUnit\decade{decade}

\tikzstyle{circleblock} = [draw, fill=white, circle, node distance=0.5cm]
\tikzstyle{block} = [draw, fill=white, rectangle, 
    minimum height=3em, minimum width=6em]
\tikzstyle{matrix} = [draw, fill=white, rectangle, 
    minimum height=1em, minimum width=1em]
\tikzstyle{sum} = [draw, fill=white, circle, node distance=1cm]
\tikzstyle{amp} = [regular polygon, regular polygon sides=3, shape border rotate=-90, draw]
\tikzstyle{input} = [coordinate]
\tikzstyle{output} = [coordinate]
\tikzstyle{pinstyle} = [pin edge={to-,thin,black}]
\tikzstyle{int}=[draw, fill=blue!20, minimum size=2em]
\tikzstyle{init} = [pin edge={to-,thin,black}]


\begin{document}
\title{EECS225A Course Notes}
\author{Anmol Parande}
\date{Fall 2021 - Professor Jiantao Jiao}
\maketitle
\textbf{Disclaimer: }These notes reflect EECS225A when I took the course (Fall 2021). They may not accurately reflect current course content, so use at your own risk.
If you find any typos, errors, etc, please raise an issue on the \href{https://github.com/parandea17/BerkeleyNotes}{GitHub repository}.
\tableofcontents
\newpage
\section{Transforms}
\subsection{Discrete Time Fourier Transform}
The Discrete Time Fourier Transform is given by \[
	X(e^{j\omega}) = \sum_{n=-\infty}^{\infty}x[n]e^{-j\omega n}.
\]
The Inverse Discrete Time Fourier Transform is given by \[
	x[n] = \frac{1}{2\pi}\int_{-\pi}^{\pi}X(e^{j\omega})e^{j\omega n}d\omega.
\]
Since the DTFT is an infinite summation, it may or may not converge.
\begin{definition}
	A signal $x[n]$ belongs to the $l^1$ class of signals if the series converges absolutely. In other words,
	\[
		\sum_{k=-\infty}^{\infty}|x[k]| < \infty.
	\]
	\label{defn:l1-class}
\end{definition}
This class covers most real-world signals.
\begin{theorem}
	If $x[n]$ is a $l^1$ signal, then the DTFT $X(e^{j\omega})$ converges uniformly and is well-defined for every $\omega$. $X(e^{j\omega})$ is also a continuous function.
	\label{thm:l1-dtft}
\end{theorem}
\begin{definition}
	A signal $x[n]$ belongs to the $l^2$ class of signals if it is square summable. In other words,
	\[
		\sum_{k=-\infty}^{\infty}|x[k]|^2 < \infty.
	\]
	\label{defn:l2-class}
\end{definition}
The $l^2$ class contains important functions such as $\sinc$.
\begin{theorem}
	If $x[n]$ is a $l^2$ signal, then the DTFT $X(e{j\omega})$ is defined almost everywhere and only converges in the mean-squared sense:
	\[
		\lim_{N\to\infty} \int_{-\pi}^{\pi}\left|\left(\sum_{k=-N}^N x[k]e^{-j\omega n}\right) - X(\omega)\right|^2d\omega = 0
	\]
	\label{thm:l2-dtft}
\end{theorem}
Tempered distributions like the Dirac Delta function are other functions which are important for computing the DTFT, and they arise from the theory of generalized functions.
\subsection{Z-Transform}
The Z-transform is a generalized version of the DTFT and is given by \[
	X(z) = \sum_{n=-\infty}^{\infty}x[n]z^{-n}.
\]
It is a special type of series called a \textbf{Laurent Series}.
\begin{theorem}
	A Laurent Series will converge absolutely on an open annulus \[
		A = \{z | r < |z| < R \}
	\] for some $r$ and $R$.
	\label{thm:z-transform-roc}
\end{theorem}
We can compute $r$ and $R$ using the signal $x[n]$.
\[
	r = \limsup_{n\to\infty} |x[n]|^{\frac{1}{n}}, \qquad \frac{1}{R} = \limsup_{n\to\infty}|x[-n]|^{\frac{1}{n}}.
\]
\section{Hilbert Space Theory}
Complex random vectors form a Hilbert space with inner product $\ip{X}{Y} = \expect{XY^*}$.
If we have a random complex vector, then we can use Hilbert Theory in a more efficient manner by looking at the matrix of inner products.
For simplicity, we will call this the ``inner product'' of two complex vectors.
\begin{definition}
	Let the inner product between two random, complex vectors $\bs{Z_1}, \bs{Z_2}$ \[
		\ip{\bs{Z_1}}{\bs{Z_2}} = \expect{\bs{Z_1}\bs{Z_2}^*}
	\]
	\label{defn:inner-product}
\end{definition}
The ij-th entry of the matrix is simply the scalar inner product $\expect{\bs{X_i}\bs{Y_j}^*}$ where $\bs{X_i}$ and $\bs{Y_j}$ are the ith and jth entries of $\bs{X}$ and $\bs{Y}$ respectively.
This means the matrix is equivalent to the cross correlation $\bs{R}_{XY}$ between the two vectors.
We can also specify the auto-correlation $\bs{R}_X = \ip{\bs{X}}{\bs{X}}$ and auto-covariance $\bs{\Sigma}_X = \ip{\bs{X} - \expect{\bs{X}}}{\bs{X} - \expect{\bs{X}}}$.
One reason why we can think of this matrix as the inner product is because it also satisfies the properties of inner products.
In particular, it is
\begin{enumerate}
	\item Linear: $\ip{\alpha_1\bs{V_1}+\alpha_2\bs{V_2}}{\bs{u}} = \alpha_1\ip{\bs{V_1}}{u} + \alpha_2\ip{\bs{V_2}}{u}$.
	\item Reflexive: $\ip{\bs{U}}{\bs{V}} = \ip{\bs{V}}{\bs{U}}^*$.
	\item Non-degeneracy: $\ip{\bs{V}}{\bs{V}} = \bs{0} \Leftrightarrow \bs{V} = \bs{0}$.
\end{enumerate}
Since we are thinking of the matrix as an inner product, we can also think of the norm as a matrix.
\begin{definition}
	The norm of a complex random vector is given by $\|\bs{Z}\|^2 = \ip{\bs{Z}}{\bs{Z}}$.
	\label{defn:norm}
\end{definition}
Since we are thinking of inner products as matrices instead of scalars, we can rewrite the Hilbert Projection Theorem to use matrices instead.
\begin{theorem}[Hilbert Projection Theorem]
	The minimization problem $\min_{\hat{\bs{X}}(\bs{Y})}\|\hat{\bs{X}}(\bs{Y}) - \bs{X}\|^2$ has a unique solution which is a linear function of $\bs{Y}$. The error is orthogonal to the linear subspace of $\bs{Y}$ (i.e $\ip{\bs{X} - \hat{\bs{X}}}{\bs{Y}} = \bs{0}$)
	\label{thm:hilbert-projection}
\end{theorem}
When we do a minimization over a matrix, we are minimizing it in a PSD sense, so for any other linear function $\bs{X}'$, \[
	\|\bs{X}-\hat{\bs{X}}\|^2  \preceq \|\bs{X} - \bs{X}'\|^2.
\]
\section{Linear Estimation}
In Linear Estimation, we are trying to estimate a random variable $\bs{X}$ using an observation $\bs{Y}$ with a llinear function of $\bs{Y}$.
If $\bs{Y}$ is finite dimensional, then we can say $\hat{\bs{X}}(\bs{Y}) = \bs{W}\bs{Y}$ where $\bs{W}$ is some matrix.
Using \cref{thm:hilbert-projection} and the orthogonality principle, we know that
\[
	\ip{\bs{X}-\bs{W}\bs{Y}}{\bs{Y}} = \bs{0} \Leftrightarrow \bs{R}_{XY} = \bs{W}\bs{R}_Y
\]
This is known as the \textbf{Normal Equation}.
If $\bs{R}_Y$ is invertible, then we can apply the inverse to find $\bs{W}$.
Otherwise, we can apply the pseudoinverse $\bs{R}_Y^\dagger$ to find $\bs{W}$, which may not be unique.
If we want to measure the quality of the estimation, since $\bs{X} = \bs{X}+(\bs{X}-\hat{\bs{X}})$,
\begin{align*}
	\|\bs{X}\|^2 &= \|\hat{\bs{X}}\|^2 + \|\bs{X} - \hat{\bs{X}}\|^2 \implies \\
	\|\bs{X}-\hat{\bs{X}}\|^2 &= \|\bs{X}\|^2 - \|\hat{\bs{X}}\|^2 = \bs{R}_X - \bs{X}_{XY}\bs{R}_Y^{-1}\bs{R}_{YX}
\end{align*}
\end{document}

