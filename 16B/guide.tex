\documentclass{article}
\usepackage{amssymb}
\usepackage{amsmath}
\usepackage{mathtools}
\newtheorem{theorem}{Theorem}
\newtheorem{definition}{Definition}
\newtheorem{corollary}{Corollary}
\newtheorem{proof}{Proof}
\begin{document}
\title{EE16B Course Notes}
\author{Anmol Parande}
\date{Spring 2019 - Professors Anant Sahai, Kris Pister, and Jaijeet Roychowdhury}
\maketitle
\section{Discrete Time Models}
Although circuits and other systems function in continuous time, computers can only act in discrete time.
This means the signals they send out to try and control the system are piecewise-continuous, and they can only sense the system at specified intervals $\delta$
At a discrete time-step $i$, the computer will measure $\vec{x}_d[i]$ and applies the control $\vec{u}[i]$ over the time-interval $[i, i+1]$.
Keep in mind that $i=\delta t$ where $t$ is the continuous time.

\begin{definition}
    The discrete time model is $\vec{x}_d[i+1]=A_d\vec{x}_d[i]+B_d\vec{u}[i]+\vec{w}[i]$ where $\vec{w}[i]$ is some disturbance.
\end{definition}

\subsection{Converting between continuous time and discrete time}
To see how to convert between a continuous time model to a discrete time model, consider the following system.
\[
    \begin{array}{c}
        \frac{dx}{dt}=\lambda x+u, u=e^{st}\\
        \left(
            \frac{d}{dt}-\lambda
            \right)x = u
    \end{array}\\
\]

Solving will give $x=\alpha e^{\lambda t}+\beta e^{st}$
Plugging this into the original equation
\[
    \begin{array}{c}
        \alpha \lambda e^{st}+\beta se^{st} = \lambda (\alpha e^{\lambda t}+\beta e^{st})+e^{st}\\
        \text{if}\> \lambda \ne s\\
        \beta s = \beta \lambda + 1 \implies \beta = \frac{1}{s-\lambda}
    \end{array}
    \]
Since the computer only outputs constant signals, $s=0$, so $x=\alpha e^{\lambda t}-\frac{u}{\lambda}$
If $x(0)=x_0$
\[
    \begin{array}{c}
        x(0)=\alpha - \frac{u}{\lambda}=x_0\\
        x = (x_0+\frac{u/\lambda})e^{\lambda t}-\frac{u}{\lambda}\\
        x = x_0 e^{\lambda t}+u\left(
            \frac{e^{\lambda t}-1}{\lambda}
        \right)
    \end{array}
\]

Thus the discrete time model is
$$x[i+1]=e^{\lambda \delta}x[i]+\left(
    \frac{e^{\lambda t}-1}{\lambda}\right) u[i]$$
since the initial condition for the next time-step is simple the value at the previous time-step

\subsection{Finding A and B}
What would happen if we didn't know what A and B were though? It turns out that we can find it by gathering data about the system and then using least squares.
In the scalar case:
\[
    \begin{array}{c}
        x[i+1]=ax[i]+bu[t]+w[i]\\
        x[1]=ax[0]+bu[0]\\
        x[2]=ax[1]+bu[1]\\
        \vdots\\
        x[m]=ax[m-1]+bu[m-1] 
    \end{array}
\]
This means with $m$ observations, we can construct a matrix $D$ and a vector $\vec{s}$ such that we have the least squares problem
\[
    D\vec{p}=\vec{s}, \vec{p}=
\left[
    \begin{array}{c}
        a\\
        b\\
    \end{array}
\right]
\]
The same thing can be done in the vector case. We just create multiple parallel systems by multiplying out the rows.
\section{Control Theory}
\subsection{Stability}
The question of stability is whethere or not disturbances will over time control a system if left alone.
Specificically, it is asking if the disturbances are bounded, will the states be bounded too? 
\begin{definition}
    Bounded Input, Bounded Output (BIBO) stability means if $|\vec{w}[i| < \epsilon$, then $\exists \delta>0$ such that $|\vec{x}[i| < \delta$
\end{definition}

To see where this definition comes into play, consider the following scalar systems:
\begin{itemize}
    \item[1.] $x[i+1] = u[i] + w[i]$ 
    \item[2.] $x[i+1] = \frac{1}{2}x[i] + u[i]+w[i]$ ("Weak" dependence on previous state)
    \item[3.] $x[i+1] = 2 x[i]+u[i]+w[i]$ ("Strong" dependenceon previous state)
\end{itemize}
\begin{itemize}
    \item[System 1:] In this system, there is no dependence on the previous state.
    Thus any disturbances are limited to affecting only the next state and no others. 
    This makes the system stable because disturbances will not add up. 
    To see this mathematically, pretend $u[i] = 0$ since we want to see how the system evolves without doing anything.
    $$|x[i+1]|=|w[i]| < \epsilon, \>\>\> \therefore |x[i]|<\epsilon$$
    Thus we have BIBO stability
    \item[System 2:] Unlike the previous system, this system has a "weak" dependence on previous states. We consider it "weak" because only half of the previous state influences the next state.
    Intuitively, this system must be BIBO stability then because although disturbances will accumulate, they will be attenuated through an infinite geometric sum. Once again, we will pretend $\vec{u}[i]=0$ to check this mathematically.
    $$|x[i+1]| = \frac{1}{2}x[i]+w[i] < \frac{1}{2}x[i]+ \epsilon$$
    Unrolling the recursive relationship, we see
    $$|x[i+1]| < \epsilon+\frac{1}{2}\epsilon+\frac{1}{4}\epsilon+...+\left( \frac{1}{2} \right)^i\epsilon$$
    $$\lim_{i->\infty}{|x[i+1]}<\frac{\epsilon}{1-\frac{1}{2}} = 2\epsilon$$
    Thus the system is BIBO stability since $|x[i]| < 2\epsilon$
    \item[System 3:] This system has a "strong" dependence on previous states since the contribution of previous states is doubled each times.
    Intuitively, this means the system must be unstable because small disturbances will be explode over time. To see this mathematically, once again consider what happens if we apply no inputs.
    $$|x[i+1]| = 2x[i]+w[i] < 2x[i]+ \epsilon$$
    Unrolling the recursion,
    $$|x[i+1]| < \epsilon+2\epsilon+4\epsilon+...+2^i\epsilon$$
    $$\lim_{i->\infty}{|x[i+1]} < \infty$$
    Thus $|x[i]|$ has no bound, so the system is not stable.
\end{itemize}

Note that in the scalar case, as long as $|\lambda| < 1$ for the system $x[i+1]=\lambda x[i]+bu[i]+w[i]$, the system will be stable. Note that this is $\lambda<1$, not $\lambda \leq 1$. That is because when $\lambda=1$, we can see the errors will add up.
It turns out that in the vector case, for the system $\vec{x}[t+1]=A \vec{x}[t]+B\vec{u}[t]+\vec{w}[t]$, the system will be stable if the eigenvalues of A are less than 1.
\begin{proof}
    If $\exists \> n \> \vec{v}_i$ such that $A\vec{v}_i=\lambda_i\vec{v}_i$, then changing coordinates to the eigenbasis gives 
    $$\tilde{A}=
        \left[
        \begin{array}{ccc}
        \lambda_1 &  & 0 \\
         & \ddots &  \\
        0 &  & \lambda_n \\
        \end{array}
        \right] $$
    This means $\vec{\tilde{x}}[t+1]=\tilde{A}\vec{\tilde{x}}[t]+V^{-1}B\vec{u}[t]+V^{-1}\vec{w}[t]$ where $V$ is the matrix whose columns are A's eigenvectors.
    This is merely n different scalar cases, but we still have to make sure that $|\vec{w}[t]|<\epsilon \implies |V^{-1}\vec{w}[t]|<\kappa$ where $\kappa$ is some bound.
    Lets say $m=V_{ij}^{-1}$ is the largest entry in $V^{-1}$. In the worst case all $\vec{w}_i[t]$ and all $V_{ij}=m$.
    Then $\forall i, \left[V^{-1}\vec{w}[t]\right]_i \leq n m \epsilon$ where $n$ is the dimension of $V$, which means $|V^{-1}\vec{w}[t]|<\kappa$
    Thus we can say that $\tilde{x}[t]$ is BIBO implies that $\vec{x}[t]$ is BIBO.
\end{proof}
\subsection{Closed Loop Control}
Lets say we have a system which is modeled by $\vec{x}[t+1]=A \vec{x}[t]+B\vec{u}[t]+\vec{w}[t]$
We must choose our inputs $\vec{u}[t]$ based on only $\vec{x}[t]$ since we can't observe the disturbances. How should we choose our inputs to control the system?
If we try $\vec{u}[t]=k\vec{x}[t]$, then
\[
    \begin{array}{c}
        \vec{x}[t+1]=A\vec{x}[t]+Bk\vec{x}[t]+\vec{w}[t]\\
        \vec{x}[t+1]=(A+Bk)\vec{x}[t]+\vec{w}[t]
    \end{array}
\] 
Which means that if we choose $K$ appropriately, we can potentially change the behavior of the system.
Consider the following simple scalar case example 
\[
    \begin{array}{c}
        x[t+1]=3x[t]+u[t]+w[t]\\
        u[t] = kx[t]\\
        x[t+1] = (3+k)x[t]+w[t]
    \end{array}{c}
\]

Notice that even if the system is initially unstable, we can set our $k$ to make it stable since the domain of $\lambda=3+k$ is $\infty$
What if we have a system $\vec{x}[t+1]=A\vec{x}[t]+\vec{b}u[t]+\vec{w}[t]$ where $A, \vec{b}$ are controllable, but we only get a scalar input $u[t]$.
Can we find a set of feedback gains $\vec{f}^T=[f_0, f_1, ..., f_{k-1}]$ such that $u[t]=-\vec{f}^T\vec{x}[t]$ makes $A-\vec{b}\vec{f}^T$ have the eigenvalues we want?
\break \break
We know how to easily solve scalar cases like $z[t+1]=az[t]+u[t]$, but it is hard to get this from a vector case because we only have a simple input, so we will be limited in how we can set our eigenvalues this way.
Since this won't work, consider the folowing recursive system.
$$z[t+1]=a_{k-1}z[t]+a_{k-2}z[t-1]+...+a_0z[t-(k-1)]+u[t]$$
This is definetly controllable since our input $u[t]$ could potentially be any value in $\mathbb{R}$. 
Moreover, if we choose $u[t]=-\sum_{i=0}^{k-1}{f_i z[t-(k-1)+i]}$, we can create any recurrence relation we want.
For example, if we wanted $z[t+1]=d_{k-1}z[t]+d_{k-2}z[t-1]+...+d_0z[t-(k-1)]$, then $f_i=a_i-d_i$
Writing this in vector form, we get
\[
    \begin{array}{c}
    \vec{z}[t+1]=A_z z[t]+\vec{b}u[t] \\
    \\
    \vec{z}[t] = \left[
        \begin{array}{c}
            z[t-(k-1)]\\
            \vdots\\
            z[t-2]\\
            z[t-1]\\
            z[t]
        \end{array}
    \right]
    \end{array}
\]
Notice that the vector $\vec{z}[t]$ here is different from the scalar $z[t]$ which it is made up of. Each "state" here is a scalar quantity, but to represent the recursive relationship properly, we collect all of the previous states in a vector to define the next state.
\break\break
Given our definition of $\vec{z}[t]$ (which are the scalar values of previous states), we can fill in values for $A_z$ and $\vec{b}$
\[
    A_z = \left[
        \begin{array}{cccccc}
            0 & 1 & 0 & 0 & ... & 0\\
            0 & 0 & 1 & 0 & ... & 0\\
            0 & 0 & 0 & 1 & ... & 0\\
            \vdots & & & & \ddots & \vdots\\
            a_0 & a_1 & a_2 & ... & & a_{k-1}
        \end{array}
    \right],\>
    \vec{b} = \left[
        \begin{array}{c}
            0\\
            0\\
            0\\
            \vdots\\
            1
        \end{array}
    \right]
\]

To understand where these come from, thing of the $A_z$ matrix "scrolling down" the state vector $\vec{z}[t]$ to get how it influences $\vec{z}[t+1]$. 
All the rows except the last are 0's and 1's since the previous states can't change, but the last row determines the next state. 
Notice that all entries in the b-vector are 0 except the last since we can't affect the past with present control.
We can assume $a_0\ne 0$ since we defined the recurrence to have length $k$. Breaking $A_z$ into its block components gives us some more insight to its structure.
\[
    A_z = \left[
        \begin{array}{c}
            \begin{array}{c|c}
                0 & I_{k-1}\\
            \end{array} \\
            \hline \\
            \vec{a}^T
        \end{array}
    \right]
\]
We can check through multiplication that this system will be controllable.
When the A matrix of a system is in this form, we say it is in \textbf{Controllable Canonical Form}.

Let's try finding the eigenvalues of $A_z$. If $A_z$ is large, it'll be hard to calculate the characteristic polynomial, so instead lets guess their form and figure them out that way.
\[
    \vec{v}_\lambda = \left[
        \begin{array}{c}
            1\\
            *\\
            *\\
            \vdots
        \end{array}
    \right]
    \]
Where * is any element.
We know $A_z\vec{v}_\lambda=\lambda v_\lambda$, and since $A_z$ scrolls the vector down, we can see that the second * must be $\lambda$. Applying this same logic to the second *, it must be $\lambda^2$.
Continuing this, we get
\[
    \vec{v}_\lambda = \left[
        \begin{array}{c}
            1\\
            \lambda\\
            \lambda^2\\
            \vdots\\
            \lambda^{k-1}
        \end{array}
    \right]
    \]
If we wanted to find the characteristic polynomial, we could look at the last entry in $A_z\vec{v}_\lambda$
\[
    \begin{array}{c}
        \sum_{i=0}^{k-1}{a_i\lambda_i}=\lambda*\lambda^{k-1}=\lambda_k\\
        \lambda_k-\sum_{i=0}^{k-1}{a_i\lambda_i}=\lambda*\lambda^{k-1} = 0
    \end{array}
    \]

Now that we know about this matrix form, the question becomes: can we transform any controllable system with matrix A into controllable canonical form?
(i.e can we make $\vec{x}[t+1]=A\vec{x}[t]+\vec{b}u[t]$ into $\vec{z}[t+1]=A_z\vec{z}[t]+\vec{b}_zu[t]$)
We know that $A\vec{b}, A^2\vec{b}, ... A^{k-1}\vec{b}$ are linearly independent since our system is controllable. Let this define basis $\mathbb{G}$. If we write the matrix A in the basis of G, we get $\tilde{A}=G^{-1}AG$
The naming choice is very appropriate here because this matrix multiplication will be painful by hand, so lets look for another way to get $\tilde{A}$.

Remember that any linear transformation is defined by what it does to the basis vectors. We know
\[
    \tilde{b} = G^{-1}\vec{b} = \left[
        \begin{array}{c}
            1\\
            0\\
            0\\
            \vdots\\
        \end{array}
    \right]
\]
since $\vec{b}$ is the first basis vector of $\mathbb{G}$. Thus,
\[
    \tilde{A}\tilde{\vec{b}}=G^{-1}A\vec{b} = \left[
        \begin{array}{c}
            0\\
            1\\
            0\\
            \vdots\\
        \end{array}
    \right],\>
    \tilde{A}^2\tilde{\vec{b}} = G^{-1}A^2\vec{b} = \left[
        \begin{array}{c}
            0\\
            0\\
            1\\
            \vdots\\
        \end{array}
    \right] ...
\]
This gives us $k$ columns of $\tilde{A}$. For the last column must be a linear combination of all the basis vectors, so $\tilde{A}^k\tilde{\vec{b}}=\sum_{i=0}^{k-1}{\alpha_i\tilde{A}^i\tilde{\vec{b}}}$
This means
\[
    \tilde{A} = \left[
        \begin{array}{cccc}
            0 & 0 & ... & \alpha_0\\
            1 & 0 & ... & \alpha_1\\
            0 & 1 & ... & \alpha_2\\
            \vdots & & \ddots & \vdots\\
            0 & ... & 1 & \alpha_{k-1}
        \end{array}
    \right]
    \]
which is $A_z^T$. Now we have a basis $\mathbb{G}$ with $(\tilde{A}, \tilde{b}$), but we want a basis $\mathbb{H}$ whith $(\tilde{A}^T, \vec{b}_z)$
We want $\tilde{A}^T, \vec{b}_z$ to be controllable, which means $\vec{b}_z, \tilde{A}^T\vec{b}_z,...,(\tilde{A}^T)^{k-1}\vec{b}_z$ are linearly independent. 
Notice that the matrix H composed on these basis vectors will take us from basis $\mathbb{H}$ to basis $\mathbb{G}$. This means to get $(\tilde{A}^T, \vec{b}_z)$ from $(A, \vec{b})$
we can apply the change of coordinates $G^{-1}H$ so $\vec{x}_z[t]=HG^{-1}\vec{x}[t]$ and put our entire system into controllable canonical form.
\end{document}