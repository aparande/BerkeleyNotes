\documentclass[12pt]{article}

\setlength{\topmargin} {-1in}

% Style modifications
\oddsidemargin  0.25in
\evensidemargin 0.25in
\textwidth      6.0in
%\textheight     8.0in
%\textheight     8.5in % Changed on 13 September 2012
\textheight     9in % Changed on 3 October 2013
\parskip        0.1in
\parindent      0.0in
\headheight     1.0in

\headsep        .25in

\usepackage{graphicx}
\usepackage{amsmath,amsfonts,amssymb,amscd,verbatim,graphicx,fancyhdr,tikz}
\usepackage{mathtools} %for minus plus sign alignment in matrices and vectors, added 13 Feb 2017

\usepackage{pgfplots}
\usetikzlibrary{arrows,automata}
\usetikzlibrary{positioning}
\usetikzlibrary{shapes.geometric}

\usepgfplotslibrary{groupplots}
\pgfplotsset{compat=1.16}

\usepackage{tikz} 
\usetikzlibrary{arrows.meta}
\usepackage{circuitikz}

\usepackage{esdiff}
\usepackage{siunitx}

\usepackage{palatino}
\usepackage{enumerate}
\usepackage{multicol}
\usepackage{listings}
\usepackage{color}
\usepackage{float}
\usepackage{esdiff} % Derivatives
\usepackage{hyperref}
\usepackage{cleveref}
\usepackage{bookmark}
\usepackage{slashbox}
\usepackage{tabularx}
\usepackage{booktabs}
\usepackage{siunitx}

\newcolumntype{Y}{>{\centering\arraybackslash}X}

\hypersetup{
    colorlinks=true,
    linkcolor=blue,
    filecolor=magenta,      
    urlcolor=blue,
}

\newcommand{\bblue}{\color[rgb]{0.2,0.2,0.7}}

\usepackage{mcode}
%Alternative: http://perso.telecom-bretagne.eu/gillesbertrand/tutorials/#sujet44
%This latter one uses the xypackage
%\usepackage{x}

\newcommand\eqnnumber{\addtocounter{equation}{1}\tag{\theequation}} % Allows us to tag the last equation in an align

\usepackage{accents}
\newcommand*{\dt}[1]{%
\accentset{\mbox{\large\bfseries .}}{#1}}
\newcommand*{\ddt}[1]{%
\accentset{\mbox{\large\bfseries .\hspace{-0.25ex}.}}{#1}}
\newcommand{\B}[1]{\mathbf{#1}}
\DeclareMathAlphabet{\V}{OML}{cmm}{b}{it}


% To get decent pdf files:
\usepackage{caption}
\usepackage{subcaption}

\newcommand{\bs}[1]{\boldsymbol{#1}}
\newcommand{\underscore}{\char`\_}
\newcommand{\tildetext}{\char`\~}
\newcommand{\defn}{\stackrel{\triangle}{=}}
\newcommand{\deriv}[2]{\frac{d{#1}}{d{#2}}}
\newcommand{\cas}{\text{cas }}

\lstset{columns=fullflexible,
    xleftmargin=0pt,
    frame=single,
    keepspaces=false,
    tabsize=2,
    breaklines=true,%
    morekeywords={matlab2tikz},
    keywordstyle=\color{blue},%
    morekeywords=[2]{1}, keywordstyle=[2]{\color{black}},
    identifierstyle=\color{black},%
    showstringspaces=false,%without this there will be a symbol in the places where there is a space
    numbers=left,%
    numberstyle={\tiny \color{black}},% size of the numbers
    numbersep=5pt, % this defines how far the numbers are from the text
    emph=[1]{for,end,break},emphstyle=[1]\color{red}, %some words to emphasise
    %emph=[2]{word1,word2}, emphstyle=[2]{style},
}

\newtheorem{theorem}{Theorem}
\newtheorem{definition}{Definition}
\newtheorem{corollary}{Corollary}
\newtheorem{proof}{Proof}

\DeclareMathOperator*{\argmin}{argmin}
\DeclareMathOperator*{\argmax}{argmax}

\newcommand{\prob}[1]{\mathbb{P}\left[#1\right]}
\newcommand{\expect}[1]{\mathbb{E}\left[#1\right]}
\newcommand{\var}[1]{Var\left(#1\right)}
\newcommand{\cov}[1]{Cov\left(#1\right)}
\newcommand{\normal}[3]{#1\sim\mathcal{N}\left(#2, #3\right)}
\newcommand{\norm}[1]{\|#1\|}
\newcommand{\vecnorm}[1]{\|\vec{#1}\|}
\newcommand{\iid}{independently and identically distributed }
\renewcommand{\arraystretch}{1.5}

\DeclareSIUnit\decade{decade}

\tikzstyle{circleblock} = [draw, fill=white, circle, node distance=0.5cm]
\tikzstyle{block} = [draw, fill=white, rectangle, 
    minimum height=3em, minimum width=6em]
\tikzstyle{matrix} = [draw, fill=white, rectangle, 
    minimum height=1em, minimum width=1em]
\tikzstyle{sum} = [draw, fill=white, circle, node distance=1cm]
\tikzstyle{amp} = [regular polygon, regular polygon sides=3, shape border rotate=-90, draw]
\tikzstyle{input} = [coordinate]
\tikzstyle{output} = [coordinate]
\tikzstyle{pinstyle} = [pin edge={to-,thin,black}]
\tikzstyle{int}=[draw, fill=blue!20, minimum size=2em]
\tikzstyle{init} = [pin edge={to-,thin,black}]


\begin{document}
\title{EE222 Course Notes}
\author{Anmol Parande}
\date{Spring 2022 - Professors Shankar Shastry and Koushil Srinath}
\maketitle
\textbf{Disclaimer: }These notes reflect EE222 when I took the course (Fall 2021). They may not accurately reflect current course content, so use at your own risk.
If you find any typos, errors, etc, please raise an issue on the \href{https://github.com/parandea17/BerkeleyNotes}{GitHub repository}.
\tableofcontents
\newpage
\section{Real Analysis}
\begin{definition}
	The extended real line is the set \[
		\{-\infty\} \cup \R \cup \{\infty\}
	\]
	\label{defn:extended-real-line}
\end{definition}
\begin{definition}
	The supremum of a set $S \subset \R$ is a value $a \in \R_e$
	such that $\forall s\in S,\ s \leq a$ and if $b \in \R_e$ such that
	$\forall s\in S,\ s \leq b$, then $a \leq b$.
	\label{defn:supremum}
\end{definition}
Supremum is essentially the ``least upper bound'' in a set. It always exists,
and is called $\sup S$. The opposite of supremum is the infinimum.
\begin{definition}
	The infinimum of a set $S \subset \R$ is a value $a \in \R_e$
	such that $\forall s\in S,\ s \geq a$ and if $b \in \R_e$ such that
	$\forall s\in S,\ s \geq b$, then $a \geq b$.
	\label{defn:infinimum}
\end{definition}
The infinimum is the ``greatest upper bound''. Like the supremum, it always
exists, and it is denoted $\inf S$. Supremum and Infinimum can be applied to
scalar function $f: S\to \R$ by letting \[
	\sup_{x\in S} f(x) = \sup \{f(x) | x\in S \}.
\]
\subsection{Norms}
\begin{definition}
	Let $V$ be a vector space of $\R$, then $\|\cdot\|: V \to \R$ is a norm if
	$\forall \bs{x},\bs{y}\in V, \alpha \in \R$, \[
		\|\bs{x}\| \geq 0 \qquad \bs{x} = 0 \Leftrightarrow \|\bs{x}\| = 0 \qquad
		\|\alpha \bs{x}\| = |\alpha|\|\bs{x}\| \qquad \|\bs{x} + \bs{y}\| \leq
		\|\bs{x}\| + \|\bs{y}\|
	\]
	\label{defn:norm}
\end{definition}
\begin{definition}
	A normed space $(V, \|\cdot\|)$ is a vector space which is equipped with a
	norm $\|\cdot\|: V \to \R$.
	\label{defn:normed-space}
\end{definition}
If we have an operator $A$ which takes as inputs vectors from normed space $(X,
\|\cdot\|_X)$ and outputs vectors in normed space $(Y, \|\cdot\|_Y)$, then we
can define another norm on the vector space of operators from $X\to Y$.
\begin{definition}
	Let $A:X\to Y$ be an operator between normed spaces $(X, \|\cdot\|_X)$ and
	$(Y, \|\cdot\|_Y)$, then the induced norm of $A$ is \[
		\|A\|_i = \sup_{\|\bs{x} \neq 0\|_X} \frac{\|A\bs{x}\|_Y}{\|\bs{x}\|_X}
	\]
	\label{defn:induced-norm}
\end{definition}
The induced norm can be thought of as the maximum gain of the operator.
\begin{definition}
	Two norms $\|\cdot\|$ and $|||\cdot|||$ on a vector space $V$ if $\exists k_1,
	k_2 > 0$ suh that \[
		\forall \bs{x}\in V,\ k_1\|\bs{x}\| \leq |||\bs{x}||| \leq k_2\|\bs{x}\|
	\]
	\label{defn:equivalent-norm}
\end{definition}
If $V$ is a finite dimensional vector space if and only if all norms of $V$ are
equivalent.

\subsection{Sets}
\begin{definition}
	Let $(V, \|\cdot\|)$ be a normed space, $a\in \R$, $a > 0$, $\bs{x}_0\in V$,
	then the open ball of radius $a$ centered around $x_0$ is given by \[
		B_a(\bs{x}_0) = \{ \bs{x} \in V \ | \ \|\bs{x} - \bs{x}_0\| < a \}
	\]
	\label{defn:open-ball}
\end{definition}
\begin{definition}
	A set $S\subset V$ is open if $\forall \bs{s}_0\in V,\ \exists \epsilon > 0$
	such that $B_\epsilon(\bs{s}_0) \subset S$.
	\label{defn:open-set}
\end{definition}
Open sets have a boundary which is not included in the set. By convention, we
say that the empty set is open.

The opposite of an open set is a closed set.
\begin{definition}
	A set $S$ is closed if $\sim S$ is open.
	\label{defn:closed-set}
\end{definition}
Closed sets have a boundary which is included in the set.

\subsection{Convergence}
\begin{definition}
	A sequence of points $\bs{x}_k$ in normed space $(V, \|\cdot\|)$ converges to
	a point $\bar{\bs{x}}$ if \[
		\forall \epsilon > 0,\ \exists N < \infty,\ \text{ such that } \forall k
		\geq N, \|\bs{x}_k - \bar{\bs{x}}\| < \epsilon
	\]
	\label{defn:convergence}
\end{definition}
Convergence means that we can always find a finite time such that after that
time, all points in the sequence stay within a specicied norm ball.
\begin{definition}
	A sequence $\bs{x}_k$ is cauchy if \[
		\forall \epsilon > 0,\ \exists N < \infty \text{ such that } \forall n,m
		\geq N, \|\bs{x}_m - \bs{x}_n\| < \epsilon
	\]
	\label{defn:cauchy-sequence}
\end{definition}
A Cauchy sequence has a looser type of convergence than a convergent sequence
since it only requires all elements to in the sequence to be part of the same
norm ball after some time instead of requiring the sequence to get closed and
closer to a single point.
\begin{theorem}
	If $\bs{x}_n$ is a convergent sequence, then $\bs{x}_n$ is a also a Cauchy
	sequence.
	\label{thm:cauchy-convergence}
\end{theorem}
\begin{definition}
	A normed space $(V, \|\cdot\|)$ is complete if every Cauchy sequence converges
	to a point in $V$
	\label{defn:complete-space}
\end{definition}
Because a complete space requires that Cauchy sequences converge, all cauchy
sequences are convergent in a complete space. Two important complete spaces are
\begin{enumerate}
	\item Every finite dimensional vector space
	\item $(C[a,b], \|\cdot\|_\infty)$, the set of continuously differentiable
		functions on the closed interval $[a,b]$ equipped with the infinity norm.
\end{enumerate}
A complete normed space is also called a \textbf{Banach Space}.

\subsection{Contractions}
\begin{definition}
	A point $x^*$ is a fixed point of a function $P:X\to X$ if $P(x^*)=x^*$.
	\label{defn:fixed-point}
\end{definition}
\begin{definition}
	A function $P:X\to X$ is a contract if there exists a constant $0\leq c < 1$
	such that \[
		\forall \bs{x},\bs{y}\in X,\ \|P(\bs{x}) - P(\bs{y})\| \leq c
		\|\bs{x}-\bs{y}\|
	\]
	\label{defn:contraction}
\end{definition}
Informally, a contraction is a function which makes distances smaller 

Suppose we look at a sequence defined by iterates of a function \[
	\bs{x}_{k+1} = P(\bs{x}_k).
\]
where $P$ is a function $P:X\to X$. When does this sequence converge, and to
what point will it converge?
\begin{theorem}[Contraction Mapping Theorem]
	If $P:X\to X$ is a contraction on the Banach space $(X, \|\cdot\|)$, then
	there is a unique $\bs{x}^*\in X$ such that $P(\bs{x}^*) = \bs{x}^*$ and
	$\forall x_0\in X$, the sequence $\bs{x}_{n+1} = P(\bs{x}_n)$ is converges to
	$\bs{x}^*$.
	\label{thm:contraction-mapping}
\end{theorem}
The contraction mapping theorem that contractions will have a unique fixed
points and that repeatedly applying the contraction will converge to the fixed
point.

\subsection{Continuity}
\begin{definition}
	A function $h:V\to W$ on normed spaces $(V, \|\cdot\|_V)$ and $(W,
	\|\cdot\|_W)$ is continuous at a point $\bs{x}_0$ if $\forall \epsilon > 0,
	\exists \delta > 0$ such that \[
		\|\bs{x}-\bs{x}\|_V < \delta \implies \|h(\bs{x}) - h(\bs{x_0})\|_W < \epsilon
	\]
	\label{thm:continuity}
\end{definition}
Continuity essentially means that given an $\epsilon-$ball in $W$, we can find a
$\delta-$ball in $V$ which is mapped to the ball in $W$.
If a function is continuous at all points $\bs{x}_0$, then we say the function
is continuous.

We can make the definition of continuity more restrictive by restraining the
rate of growth of the function.
\begin{definition}
	A function $h:V\to W$ on normed spaces $(V, \|\cdot\|_V)$ and $(W,
	\|\cdot\|_W)$ is Lipschitz continuous at $\bs{x}_0\in V$ if $\exists r > 0$
	and $L < \infty$ such that \[
		\forall \bs{x}, \bs{y}\in B_r(\bs{x}_0),\ \|h(\bs{x}) - h(\bs{y})\|_W \leq L
		\|\bs{x} - \bs{y}\|_V
	\]
	\label{defn:lipschitz-continuity}
\end{definition}
A good interpretation of Lipschitz Continuity is that given two points in a ball
around $\bs{x}_0$, the slope of the line connecting those two points is less
than $L$. It means that the function is growings slower than linear for some
region around $\bs{x}_0$. Lipschitz continuity implies continuity. If a function
is lipschitz continuous with respect to one norm, it is also lipschitz
continuous with respect to all equivalent norms.

When the function $h$ is a function on $\R^n$ and is also
differentiable, then Lipschitz continuity is easy to determine.
\begin{theorem}
	For a differentiably function $h:\R^n\to\R^n$, \[
		\exists r>0, L < \infty, \bs{x}_0\in\R^n,\ \forall \bs{x}\in B_r(\bs{x}),
		\left\lvert\left\lvert\frac{\partial h}{\partial
		\bs{x}}\right\rvert\right\rvert_2 \leq L
	\] implies Lipschitz Continuity at $\bs{x}_0$.
	\label{thm:differentiable-lipschitz}
\end{theorem}
\begin{definition}
	A function $h:\R\to V$ is piecewise continuous if $\forall k\in \mathbb{Z}$,
	$h:[-k, k] \to V$ is continuous except at a possibly finite number of points,
	and at the points of discontinuity $t_i$, $\lim_{s\to0^+} h(t_i+s)$ and
	$\lim_{s\to0^-}$ exist and are finite. 
	\label{defn:piecewise-continuous}
\end{definition}
\section{Solutions to Nonlinear Systems}
Consider the nonlinear system \[
	\diff[]{\bs{x}}{t} = f(\bs{x}, t),\ \bs{x}(t_0) = \bs{x}_0\in \R^n.
\]
\begin{definition}
	A function $\bs{\Phi}(t)$ is a solution to $\diff[]{\bs{x}}{t} = f(x, t),\
	\bs{x}(t_0) = \bs{x}_0$ on the closed interval $[t_0, t]$ if $\bs{\Phi}(t)$
	is defined on the interval $[t_0, t]$, $\diff[]{\bs{\Phi}}{t} =
	f(\bs{\Phi}(t), t)$ on the interval $[t_0, t]$, and $\bs{\Phi}(t_0) =
	\bs{x}_0$.
	\label{defn:nonlinear-solution}
\end{definition}
We say that $\Phi(t)$ is a solution in the sense of Caratheodory if \[
	\Phi(t) = \bs{x}_0 + \int_{t_0}^t f(\bs{\Phi}(\tau), \tau)d\tau.
\]
Because the sytem is non-linear, it could potentially have no solution, one
solution, or many solutions. These solutions could exist locally, or they could
exist for all time. We might also want to know when there is a solution which
depends continuously on the initial conditions.
\begin{theorem}[Local Existence and Uniqueness]
	Given $\diff{\bs{x}}{t} = f(\bs{x}, t),\ \bs{x}(t_0) = \bs{x}_0\in\R^n$ where
	$f$ is piecewise continuous in $t$ and $\exists T>t_0$ such that $\forall t\in
	[t_0, T], f$ is $L$-Lipschitz Continuous, then $\exists \delta > 0$ such that a
	solution exists and is unique $\forall t\in [t_0, t_0 + \delta]$.
	\label{thm:local-existence}
\end{theorem}
\Cref{thm:local-existence} can be proved using the Contraction Mapping Theorem
(\cref{thm:contraction-mapping}) by finding $\delta$ such that the function
$P:C_n[t_0, t_0+\delta] \to C_n[t_0, t_0+\delta]$ given by \[
	P(\bs{\Phi})(t) = \bs{x}_0 + \int_{t_0}^{t_0+\delta} f(\bs{\Phi}(\tau),
	\tau)d\tau
\]
is a contraction under the norm $\|\bs{\Phi}\|_\infty = \sup_{t_0\leq t \leq
t_0+\delta} \|\bs{\Phi}(t)\|$.
\begin{theorem}[Global Existence and Uniqueness]
	Suppose $f(\bs{x}, t)$ is piecewise continuous in $t$ and $\forall T\in [t_0,
	\infty)$, $\exists L_T < \infty$ such that $f$ is $L_T$ Lipshitz continuous
	for all $\bs{x}, \bs{y} \in \R^n$, then the nonlinear system  has exactly one
	solution on $[t_0, T]$.
	\label{thm:global-existence}
\end{theorem}
Once we know that solutions to a nonlinear system exist, we can sometimes bound
them.
\begin{theorem}[Bellman-Gronwall Lemma]
	Suppose $\lambda\in\R$ is a constant and $\mu:[a,b]\to\R$ is continuous and
	non-negative, then for a continuous function $y:[a, b]\to\R$ \[
		y(t) \leq \lambda + \int_a^t \mu(\tau)y(\tau)d\tau \implies y(t) \leq
		\lambda \text{exp}\left(\int_a^t\mu(\tau)d\tau\right)
	\]
	\label{thm:bellman-gronwall}
\end{theorem}
Another thing we might want to do is understand how the non-linear system reacts
to changes in the initial condition.
\begin{theorem}
	Suppose the system $\diff{\bs{x}}{t} = f(\bs{x}, t),\ \bs{x}(t_0) = \bs{x}_0$
	satisfies the conditions of global uniqueness and existence. Fix $T\in[t_0,
	\infty]$ and suppose $\bs{x}(\cdot)$ and $\bs{z}(\cdot)$  are two solutions
	satisfying $\diff{\bs{x}}{t} = f(\bs{x}, t), \bs{x}(t_0) = \bs{x}_0$ and
	$\diff{\bs{z}}{t} = f(\bs{z}(t), t),\ \bs{z}(t_0)=\bs{z}_0$, then $\forall
	\epsilon > 0, \exists \delta > 0$ such that \[
		\|\bs{x}_0 - \bs{z}_0\| < \delta \implies \|\bs{x} - \bs{z}\|_{\infty} <
		\epsilon.
	\]
	\label{thm:continuous-dependence-on-ic}
\end{theorem}
\Cref{thm:continuous-dependence-on-ic} is best understood by defining a function
$\Psi:\R^n \to C_n[t_0, t]$ where $\Psi(\bs{x}_0)(t)$ returns the solution to
the system given the initial condition. If the conditions of
\Cref{thm:continuous-dependence-on-ic} are satisfied, then the function $\Psi$
will be continuous.

\section{Nonlinear System Dynamics}
Consider the non-linear system \[
	\diff[]{\bs{x}}{t} = f(\bs{x}, t).
\]
\begin{definition}
	The system is autonomous if $f(x, t)$ is not explicitly dependent on time $t$.
	\label{defn:autonomous-system}
\end{definition}
\begin{definition}
	A point $x_0$ is an equilibrium point at time $t_0$ if \[
		\forall t \geq t_0, \ f(x_0, t) = 0 
	\]
	\label{defn:equilibrium-point}
\end{definition}
\subsection{Planar Dynamical Systems}
Planar dynamical systems are those with 2 state variables.
Suppose we linearize the system $\diff[]{\bs{x}}{t} = f(\bs{x})$ at an equilibrium point.
\[
	\diff[]{\bs{x}}{t} = D_f|_{\bs{x} = \bs{x_0}}\bs{x} 
\]
Depending on the eigenvalues of $D_f$, the Jacobian, we get several cases for
how this linear system behaves. We'll let $z_1$ and $z_2$ be the eigenbasis of
the \textit{phase space}.
\begin{enumerate}
	\item The eigenvalues are real, yielding solutions $z_1 = z_1(0)e^{\lambda_1
		t}, z_2 = z_2(0)e^{\lambda_2 t}$. If we eliminate the time variable, we can
		plot the trajectories of the system.
		\[
			\frac{z_1}{z_1(0)} = \left(\frac{z_2}{z_2(0)}\right)^{\frac{\lambda_1}{\lambda_2}}
		\]
		\begin{enumerate}
			\item When $\lambda_1, \lambda_2 < 0$, all trajectories converge to the origin, so we call this a \textbf{stable node}.
			\item When $\lambda_1, \lambda_2 > 0$, all trajectories blow up, so we call this an \textbf{unstable node}.
			\item When $\lambda_1 < 0 < \lambda_2$, the trajectories will converge to
				the origin along the axis corresponding to $\lambda_1$ and diverge along
				the axis corresponding to $\lambda_2$, so we call this a \textbf{saddle node}.
		\end{enumerate}
	\item There is a single repeated eigenvalue with one eigenvector. As before,
		we can eliminate the time variable and plot the trajectories on the $z_1$,
		$z_2$ axes.
		\begin{enumerate}
			\item When $\lambda < 0$, the trajetories will converge to the origin, so
				we call it an \textbf{improper stable node}
			\item When $\lambda > 0$, the trajetories will diverge from the origin, so
				we call it an \textbf{improper unstable node}
		\end{enumerate}
	\item When there is a complex pair of eigenvalues, the linear system will have
		oscillatory behavior. The Real Jordan form of $D_f$ will look like \[
			D_f = \begin{bmatrix} \alpha & \beta \\ -\beta & \alpha \end{bmatrix}.
		\]
		The parameter $\beta$ will determine the direction of the trajectories
		(clockwise if positive).
		\begin{enumerate}
			\item When $\alpha < 0$, the trajectories will spiral towards the origin,
				so we call it a \textbf{stable focus}.
			\item When $\alpha = 0$, the trajectories will remain at a constant radius
				from the origin, so we call it a \textbf{center}.
			\item When $\alpha > 0$, the trajectories will spiral away from the
				origin, so we call it an \textbf{unstable focus}.
		\end{enumerate}
\end{enumerate}
It turns out that understanding the linear dynamics at equilibrium points can be helpful in
understanding the nonlinear dynamics near equilibrium points.
\begin{theorem}[Hartman-Grobman Theorem]
	If the linearization of a planar dynamical system $\diff{\bs{x}}{t} =
	f(\bs{x})$ at an equilibrium point $\bs{x_0}$ has no zero or purely imaginary eigenvalues, then there exists a
	homeomorphism from a neighborhood $U$ of $x_0$ into $\R^2$ which takes
	trajectories of the nonlinear system and maps them onto the linearization
	where $h(\bs{x_0}) = 0$, and the homeomorphism can be chosen to preserve the
	parameterization by time.
	\label{thm:hartman-grobman}
\end{theorem}
\Cref{thm:hartman-grobman} essentially says that the linear dynamics predict the
nonlinear dynamics around equilibria, but only for a neighborhood around the
equilibrium point. Outisde of this neighborhood, the linearization may be very
wrong.

Non-linear systems can also have periodic solutions.
\begin{definition}
	A closed orbit $\gamma$ is a trajectory of the system such that $\gamma(0) =
	\gamma(T)$ for finite $T$.
	\label{defn:closed-orbit}
\end{definition}

Suppose that we have a simply connected region $D$ (meaning $D$ cannot be
contracted to a point) and we want to know if it contains a closed orbit.
\begin{theorem}[Bendixon's Theorem]
	If $\divergence(f)$ is not identically zero in a sub-region of $D$ and does not
	change sign in $D$, then $D$ contains no closed orbits.
	\label{thm:bendixons}
\end{theorem}
\Cref{thm:bendixons} lets us rule out closed orbits from regions of
$\R^2$.
\begin{definition}
	A region $M \subset \R^2$ is positively invariant for a trajectory
	$\phi_t(\bs{x})$ if $\forall x\in M, \forall t \geq 0, \phi_t(\bs{x}) \in M$.
	\label{defn:positive-invariance}
\end{definition}
A positively invariant set essentially means that once a trajectory enters the
set, it cannot leave. That means all of the vector field lines must point inside
the set.
If we have a positively invariant region, then we can determine whether it
contains closed orbits.
\begin{theorem}[Poincare-Bendixson Theorem]
	If $M$ is a compact, positively invariant set for the flow $\phi_t(\bs{x})$,
	then if $M$ contains no equilibrium points, then $M$ has a limit cycle.
	\label{thm:poincare-bendixson}
\end{theorem}
\end{document}

