\documentclass[12pt]{article}

\setlength{\topmargin} {-1in}

% Style modifications
\oddsidemargin  0.25in
\evensidemargin 0.25in
\textwidth      6.0in
%\textheight     8.0in
%\textheight     8.5in % Changed on 13 September 2012
\textheight     9in % Changed on 3 October 2013
\parskip        0.1in
\parindent      0.0in
\headheight     1.0in

\headsep        .25in

\usepackage{graphicx}
\usepackage{amsmath,amsfonts,amssymb,amscd,verbatim,graphicx,fancyhdr,tikz}
\usepackage{mathtools} %for minus plus sign alignment in matrices and vectors, added 13 Feb 2017

\usepackage{pgfplots}
\usetikzlibrary{arrows,automata}
\usetikzlibrary{positioning}
\usetikzlibrary{shapes.geometric}

\usepgfplotslibrary{groupplots}
\pgfplotsset{compat=1.16}

\usepackage{tikz} 
\usetikzlibrary{arrows.meta}
\usepackage{circuitikz}

\usepackage{esdiff}
\usepackage{siunitx}

\usepackage{palatino}
\usepackage{enumerate}
\usepackage{multicol}
\usepackage{listings}
\usepackage{color}
\usepackage{float}
\usepackage{esdiff} % Derivatives
\usepackage{hyperref}
\usepackage{cleveref}
\usepackage{bookmark}
\usepackage{slashbox}
\usepackage{tabularx}
\usepackage{booktabs}
\usepackage{siunitx}

\newcolumntype{Y}{>{\centering\arraybackslash}X}

\hypersetup{
    colorlinks=true,
    linkcolor=blue,
    filecolor=magenta,      
    urlcolor=blue,
}

\newcommand{\bblue}{\color[rgb]{0.2,0.2,0.7}}

\usepackage{mcode}
%Alternative: http://perso.telecom-bretagne.eu/gillesbertrand/tutorials/#sujet44
%This latter one uses the xypackage
%\usepackage{x}

\newcommand\eqnnumber{\addtocounter{equation}{1}\tag{\theequation}} % Allows us to tag the last equation in an align

\usepackage{accents}
\newcommand*{\dt}[1]{%
\accentset{\mbox{\large\bfseries .}}{#1}}
\newcommand*{\ddt}[1]{%
\accentset{\mbox{\large\bfseries .\hspace{-0.25ex}.}}{#1}}
\newcommand{\B}[1]{\mathbf{#1}}
\DeclareMathAlphabet{\V}{OML}{cmm}{b}{it}


% To get decent pdf files:
\usepackage{caption}
\usepackage{subcaption}

\newcommand{\bs}[1]{\boldsymbol{#1}}
\newcommand{\underscore}{\char`\_}
\newcommand{\tildetext}{\char`\~}
\newcommand{\defn}{\stackrel{\triangle}{=}}
\newcommand{\deriv}[2]{\frac{d{#1}}{d{#2}}}
\newcommand{\cas}{\text{cas }}

\lstset{columns=fullflexible,
    xleftmargin=0pt,
    frame=single,
    keepspaces=false,
    tabsize=2,
    breaklines=true,%
    morekeywords={matlab2tikz},
    keywordstyle=\color{blue},%
    morekeywords=[2]{1}, keywordstyle=[2]{\color{black}},
    identifierstyle=\color{black},%
    showstringspaces=false,%without this there will be a symbol in the places where there is a space
    numbers=left,%
    numberstyle={\tiny \color{black}},% size of the numbers
    numbersep=5pt, % this defines how far the numbers are from the text
    emph=[1]{for,end,break},emphstyle=[1]\color{red}, %some words to emphasise
    %emph=[2]{word1,word2}, emphstyle=[2]{style},
}

\newtheorem{theorem}{Theorem}
\newtheorem{definition}{Definition}
\newtheorem{corollary}{Corollary}
\newtheorem{proof}{Proof}

\DeclareMathOperator*{\argmin}{argmin}
\DeclareMathOperator*{\argmax}{argmax}

\newcommand{\prob}[1]{\mathbb{P}\left[#1\right]}
\newcommand{\expect}[1]{\mathbb{E}\left[#1\right]}
\newcommand{\var}[1]{Var\left(#1\right)}
\newcommand{\cov}[1]{Cov\left(#1\right)}
\newcommand{\normal}[3]{#1\sim\mathcal{N}\left(#2, #3\right)}
\newcommand{\norm}[1]{\|#1\|}
\newcommand{\vecnorm}[1]{\|\vec{#1}\|}
\newcommand{\iid}{independently and identically distributed }
\renewcommand{\arraystretch}{1.5}

\DeclareSIUnit\decade{decade}

\tikzstyle{circleblock} = [draw, fill=white, circle, node distance=0.5cm]
\tikzstyle{block} = [draw, fill=white, rectangle, 
    minimum height=3em, minimum width=6em]
\tikzstyle{matrix} = [draw, fill=white, rectangle, 
    minimum height=1em, minimum width=1em]
\tikzstyle{sum} = [draw, fill=white, circle, node distance=1cm]
\tikzstyle{amp} = [regular polygon, regular polygon sides=3, shape border rotate=-90, draw]
\tikzstyle{input} = [coordinate]
\tikzstyle{output} = [coordinate]
\tikzstyle{pinstyle} = [pin edge={to-,thin,black}]
\tikzstyle{int}=[draw, fill=blue!20, minimum size=2em]
\tikzstyle{init} = [pin edge={to-,thin,black}]


\begin{document}
\title{EE222 Course Notes}
\author{Anmol Parande}
\date{Spring 2022 - Professors Shankar Shastry and Koushil Srinath}
\maketitle
\textbf{Disclaimer: }These notes reflect EE222 when I took the course (Spring
2022). They may not accurately reflect current course content, so use at your own risk.
If you find any typos, errors, etc, please raise an issue on the \href{https://github.com/parandea17/BerkeleyNotes}{GitHub repository}.
\tableofcontents
\newpage
\section{Real Analysis}
\begin{definition}
	The extended real line is the set \[
		\{-\infty\} \cup \R \cup \{\infty\}.
	\]
	\label{defn:extended-real-line}
\end{definition}
\begin{definition}
	The supremum of a set $S \subset \R$ is a value $a \in \R_e$
	such that $\forall s\in S,\ s \leq a$ and if $b \in \R_e$ such that
	$\forall s\in S,\ s \leq b$, then $a \leq b$.
	\label{defn:supremum}
\end{definition}
Supremum is essentially the ``least upper bound'' in a set. It always exists,
and is called $\sup S$. The opposite of supremum is the infinimum.
\begin{definition}
	The infinimum of a set $S \subset \R$ is a value $a \in \R_e$
	such that $\forall s\in S,\ s \geq a$ and if $b \in \R_e$ such that
	$\forall s\in S,\ s \geq b$, then $a \geq b$.
	\label{defn:infinimum}
\end{definition}
The infinimum is the ``greatest upper bound''. Like the supremum, it always
exists, and it is denoted $\inf S$. Supremum and Infinimum can be applied to
scalar function $f: S\to \R$ by letting \[
	\sup_{x\in S} f(x) = \sup \{f(x) | x\in S \}.
\]
\subsection{Norms}
\begin{definition}
	Let $V$ be a vector space of $\R$, then $\|\cdot\|: V \to \R$ is a norm if
	$\forall \bs{x},\bs{y}\in V, \alpha \in \R$, \[
		\|\bs{x}\| \geq 0, \qquad \bs{x} = 0 \Leftrightarrow \|\bs{x}\| = 0, \qquad
		\|\alpha \bs{x}\| = |\alpha|\|\bs{x}\|, \qquad \|\bs{x} + \bs{y}\| \leq
		\|\bs{x}\| + \|\bs{y}\|.
	\]
	\label{defn:norm}
\end{definition}
\begin{definition}
	A normed space $(V, \|\cdot\|)$ is a vector space which is equipped with a
	norm $\|\cdot\|: V \to \R$.
	\label{defn:normed-space}
\end{definition}
If we have an operator $A$ which takes vectors from normed space $(X,
\|\cdot\|_X)$ and outputs vectors in normed space $(Y, \|\cdot\|_Y)$, then we
can define another norm on the vector space of operators from $X\to Y$.
\begin{definition}
	Let $A:X\to Y$ be an operator between normed spaces $(X, \|\cdot\|_X)$ and
	$(Y, \|\cdot\|_Y)$, then the induced norm of $A$ is \[
		\|A\|_i = \sup_{\|\bs{x}\|_X \neq 0} \frac{\|A\bs{x}\|_Y}{\|\bs{x}\|_X}
	\]
	\label{defn:induced-norm}
\end{definition}
The induced norm can be thought of as the maximum gain of the operator.
\begin{definition}
	Two norms $\|\cdot\|$ and $|||\cdot|||$ on a vector space $V$ are said to be
	equivalent if $\exists k_1,
	k_2 > 0$ such that \[
		\forall \bs{x}\in V,\ k_1\|\bs{x}\| \leq |||\bs{x}||| \leq k_2\|\bs{x}\|
	\]
	\label{defn:equivalent-norm}
\end{definition}
If $V$ is a finite dimensional vector space if and only if all norms of $V$ are
equivalent.

\subsection{Sets}
\begin{definition}
	Let $(V, \|\cdot\|)$ be a normed space, $a\in \R$, $a > 0$, $\bs{x}_0\in V$,
	then the open ball of radius $a$ centered around $x_0$ is given by \[
		B_a(\bs{x}_0) = \{ \bs{x} \in V \ | \ \|\bs{x} - \bs{x}_0\| < a \}
	\]
	\label{defn:open-ball}
\end{definition}
\begin{definition}
	A set $S\subset V$ is open if $\forall \bs{s}_0\in S,\ \exists \epsilon > 0$
	such that $B_\epsilon(\bs{s}_0) \subset S$.
	\label{defn:open-set}
\end{definition}
Open sets have a boundary which is not included in the set. By convention, we
say that the empty set is open.

The opposite of an open set is a closed set.
\begin{definition}
	A set $S$ is closed if $\sim S$ is open.
	\label{defn:closed-set}
\end{definition}
Closed sets have a boundary which is included in the set.

\subsection{Convergence}
\begin{definition}
	A sequence of points $\bs{x}_k$ in normed space $(V, \|\cdot\|)$ converges to
	a point $\bar{\bs{x}}$ if \[
		\forall \epsilon > 0,\ \exists N < \infty,\ \text{ such that } \forall k
		\geq N, \|\bs{x}_k - \bar{\bs{x}}\| < \epsilon
	\]
	\label{defn:convergence}
\end{definition}
Convergence means that we can always find a finite time such that after that
time, all points in the sequence stay within a specified norm ball.
\begin{definition}
	A sequence $\bs{x}_k$ is cauchy if \[
		\forall \epsilon > 0,\ \exists N < \infty \text{ such that } \forall n,m
		\geq N, \|\bs{x}_m - \bs{x}_n\| < \epsilon
	\]
	\label{defn:cauchy-sequence}
\end{definition}
A Cauchy sequence has a looser type of convergence than a convergent sequence
since it only requires all elements to in the sequence to be part of the same
norm ball after some time instead of requiring the sequence to get closer and
closer to a single point.
\begin{theorem}
	If $\bs{x}_n$ is a convergent sequence, then $\bs{x}_n$ is a also a Cauchy
	sequence.
	\label{thm:cauchy-convergence}
\end{theorem}
\begin{definition}
	A normed space $(V, \|\cdot\|)$ is complete if every Cauchy sequence converges
	to a point in $V$.
	\label{defn:complete-space}
\end{definition}
Because a complete space requires that Cauchy sequences converge, all cauchy
sequences are convergent in a complete space. Two important complete spaces are
\begin{enumerate}
	\item Every finite dimensional vector space
	\item $(C[a,b], \|\cdot\|_\infty)$, the set of continuously differentiable
		functions on the closed interval $[a,b]$ equipped with the infinity norm.
\end{enumerate}
A complete normed space is also called a \textbf{Banach Space}.

\subsection{Contractions}
\begin{definition}
	A point $\bs{x}^*$ is a fixed point of a function $P:X\to X$ if
	$P(\bs{x}^*)=\bs{x}^*$.
	\label{defn:fixed-point}
\end{definition}
\begin{definition}
	A function $P:X\to X$ is a contraction if $\exists c\in\R, 0 \leq c < 1$ 
	such that \[
		\forall \bs{x},\bs{y}\in X,\ \|P(\bs{x}) - P(\bs{y})\| \leq c
		\|\bs{x}-\bs{y}\|
	\]
	\label{defn:contraction}
\end{definition}
Informally, a contraction is a function which makes distances smaller. Suppose we look at a sequence defined by iterates of a function \[
	\bs{x}_{k+1} = P(\bs{x}_k)
\]
where $P$ is a function $P:X\to X$. When does this sequence converge, and to
what point will it converge?
\begin{theorem}[Contraction Mapping Theorem]
	If $P:X\to X$ is a contraction on the Banach space $(X, \|\cdot\|)$, then
	there is a unique $\bs{x}^*\in X$ such that $P(\bs{x}^*) = \bs{x}^*$ and
	$\forall \bs{x}_0\in X$, the sequence $\bs{x}_{n+1} = P(\bs{x}_n)$ converges to
	$\bs{x}^*$.
	\label{thm:contraction-mapping}
\end{theorem}
The contraction mapping theorem proves that contractions have a unique fixed
points, and that repeatedly applying the contraction will converge to the fixed
point.

\subsection{Continuity}
\begin{definition}
	A function $h:V\to W$ on normed spaces $(V, \|\cdot\|_V)$ and $(W,
	\|\cdot\|_W)$ is continuous at a point $\bs{x}_0$ if $\forall \epsilon > 0,
	\exists \delta > 0$ such that \[
		\|\bs{x}-\bs{x}_0\|_V < \delta \implies \|h(\bs{x}) - h(\bs{x_0})\|_W < \epsilon
	\]
	\label{thm:continuity}
\end{definition}
Continuity essentially means that given an $\epsilon-$ball in $W$, we can find a
$\delta-$ball in $V$ which is mapped to the ball in $W$.
If a function is continuous at all points $\bs{x}_0$, then we say the function
is continuous.

We can make the definition of continuity more restrictive by restraining the
rate of growth of the function.
\begin{definition}
	A function $h:V\to W$ on normed spaces $(V, \|\cdot\|_V)$ and $(W,
	\|\cdot\|_W)$ is Lipschitz continuous at $\bs{x}_0\in V$ if $\exists r > 0$
	and $L < \infty$ such that \[
		\forall \bs{x}, \bs{y}\in B_r(\bs{x}_0),\ \|h(\bs{x}) - h(\bs{y})\|_W \leq L
		\|\bs{x} - \bs{y}\|_V
	\]
	\label{defn:lipschitz-continuity}
\end{definition}
A good interpretation of Lipschitz Continuity is that given two points in a ball
around $\bs{x}_0$, the slope of the line connecting those two points is less
than $L$. It means that the function is growing slower than linear for some
region around $\bs{x}_0$. Lipschitz continuity implies continuity. If a function
is lipschitz continuous with respect to one norm, it is also lipschitz
continuous with respect to all equivalent norms.

When the function $h$ is a function on $\R^n$ and is also
differentiable, then Lipschitz continuity is easy to determine.
\begin{theorem}
	For a differentiable function $h:\R^n\to\R^n$, \[
		\exists r>0, L < \infty, \bs{x}_0\in\R^n,\ \forall \bs{x}\in B_r(\bs{x}_0),
		\left\lvert\left\lvert\frac{\partial h}{\partial
		\bs{x}}\right\rvert\right\rvert_2 \leq L
	\] implies Lipschitz Continuity at $\bs{x}_0$.
	\label{thm:differentiable-lipschitz}
\end{theorem}
This captures the idea of growing slower than linear in high dimensional space.
\begin{definition}
	A function $h:\R\to V$ is piecewise continuous if $\forall k\in \mathbb{Z}$,
	$h:[-k, k] \to V$ is continuous except at a possibly finite number of points,
	and at the points of discontinuity $t_i$, $\lim_{s\to0^+} h(t_i+s)$ and
	$\lim_{s\to0^-}h(t_i+s)$ exist and are finite. 
	\label{defn:piecewise-continuous}
\end{definition}
\section{Differential Geometry}
\begin{definition}
	$M\subset \R^n$ is a $m$-dimensional smooth sub-manifold of $\R^n$ if $\forall
	\bs{p}\in M,\ \exists r > 0$ and $F: B_r(\bs{p}) \to \R^{n-m}$ such that \[
		\begin{aligned}
			M \cap B_r(\bs{p}) = \{\bs{x}\in B_r(\bs{p}) | F(\bs{x}) = 0\},\\
			F\text{ is smooth,}\\
			\forall \bar{\bs{x}} \in M \cap B_r(\bs{p}),
			\text{Rank}\left(\frac{\partial F}{\partial \bs{x}}
			\bigg\rvert_{\bar{\bs{x}}}\right) = n - m
		\end{aligned}
	\]
	\label{defn:manifold}
\end{definition}
By \cref{defn:manifold}, a manifold is essentially defined as the 0-level set of
some smooth function $F$ and can be thought of as a surface embedded in a higher
dimension.
\begin{definition}
	The tangent space of a manifold $M$ at $\bs{p}\in M$ is given by \[
		T_{\bs{p}}M = \text{Null}\left(\frac{\partial F}{\partial
		\bs{x}}\bigg |_{\bs{p}}\right)
	\]
	\label{defn:tangent-space}
\end{definition}
The tangent space consists of all vectors tangent to the manifold at a
particular point $\bs{p}$.
\begin{definition}
	The Tangent Bundle of a manifold $M$ is the collection of all tangent spaces \[
		T_M = \bigcup_{\bs{p}\in M} T_{\bs{p}} M
	\]
	\label{defn:tangent-bundle}
\end{definition}
\begin{definition}
	A vector field $f:M\to T_M$ on a manifold $M$ is an assignment of each point $\bs{p}\in
	M$ to a vector in the tangent space in that point $T_{\bs{p}}M$.
	\label{defn:vector-field}
\end{definition}
Therefore, a vector field can be thought of as a curve through the tangent
bundle of a manifold.
\begin{definition}
	The Lie Derivative of a function $V$ with respect to a vector field $f$ is
	given by \[
			L_fV = (\nabla_{\bs{x}}V)^\top f(\bs{x}).
	\]
	\label{defn:lie-derivative}
\end{definition}
A Lie Derivative is essentially a directional derivative, and it measures how a
function changes along a vector field.
\begin{definition}
	Suppose that $f(\bs{x})$ and $g(\bs{x})$ are vector fields. The Lie Bracket of
	$f$ and $g$ is given by \[
		[f, g] = L_fg - L_gf
	\]
	\label{defn:lie-bracket}
\end{definition}
The Lie Bracket is another vector field, and it essentially measures the
difference between moving along vector field $f$ and vector field $g$ across
some infinitesimal distance. Another way to think about the Lie Bracket is as a
measure of the extent to which $f$ and $g$ commute with each other. The Lie
Bracket is also sometimes denoted using the adjoint map \[
	\ad_fg = [f, g].
\]
It is helpful when chaining Lie Brackets since we can denote \[
	[f,[f,[f,\cdots[f,g]]]] = \ad_f^ig.
\]
Since the Lie Bracket is a vector field, we can look at Lie Derivatives with
respect to the Lie Bracket of two vector fields.
\begin{theorem}
	For a function $h$ and vector fields $f$ and $g$, \[
		L_{[f,g]}h = L_fL_gh - L_gL_fh
	\]
	\label{thm:lie-brack-derivative}
\end{theorem}
We can also use relate repeated Lie Derivatives to doing repeated Lie Brackets.
\begin{theorem}
	\[
		L_gL_f^ih(\bs{x}) = 0 \Leftrightarrow L_{\ad_f^ig}h(\bs{x}) = 0
	\]
	\label{thm:repeated-lie}
\end{theorem}
\begin{definition}
	Suppose $f_1,f_2,\cdots,f_n$ are vector fields. A distribution $\Delta$ is the
	span of the vector fields at each point $\bs{x}$: \[
		\Delta(\bs{x}) = \text{span}\{f_1(\bs{x}), f_2(\bs{x}),\cdots,f_n(\bs{x})\}.
	\]
	\label{defn:distribution}
\end{definition}
At each point $\bs{x},\ \Delta(\bs{x})$ is a subspace of the tangent space at
$\bs{x}$.
\begin{definition}
	The dimension of a distribution at a point $\bs{x}$ is given by \[
		\text{Dim }\Delta(\bs{x}) = \text{Rank}\left(\begin{bmatrix}
				f_1(\bs{x}) & \bigg\lvert & f_2(\bs{x}) & \bigg\lvert & \cdots &
				\bigg\lvert & f_n(\bs{x})
		\end{bmatrix}\right)
	\]
	\label{defn:distribution-dimension}
\end{definition}
Distributions have different properties which are important to look at.
\begin{definition}
	A distribution $\Delta$ is nonsingular, also known as regular, if its
	dimension is constant.
	\label{defn:regular-dimension}
\end{definition}
\begin{definition}
	A distribution $\Delta$ is involutive if \[
		\forall f, g\in \Delta, \quad [f, g] \in \Delta
	\]
	\label{defn:involutive-distribution}
\end{definition}
In involutive distributions, you can never leave the distribution by traveling
along vectors inside the distribution.
\begin{definition}
	A nonsingular $K$-dimensional distribution $\Delta(\bs{x}) =
	\text{span}\{f_1(\bs{x}), \cdots, f_k(\bs{x})\}$ is completely integrable if
	$\exists \phi_1,\cdots,\phi_{n-k}$ such that $\forall i,k,\ L_{f_k}\phi_i = 0$
	and $\nabla_{\bs{x}}\phi_i$ are linearly independent.
	\label{thm:involutive}
\end{definition}
It turns out that integrability and involutivity are equivalent to each other.
\begin{theorem}[Frobenius Theorem]
	A nonsingular $\Delta$ is completely integrable if and only if $\Delta$ is
	involutive.
	\label{thm:frobenius}
\end{theorem}
\section{Nonlinear System Dynamics}
Consider the nonlinear system \[
	\diff[]{\bs{x}}{t} = f(\bs{x}, t).
\]
$f$ is a vector field which potentially changes with time and governs how the
system evolves.
\begin{definition}
	The system is autonomous if $f(\bs{x}, t)$ is not explicitly dependent on time $t$.
	\label{defn:autonomous-system}
\end{definition}
\begin{definition}
	A point $x_0$ is an equilibrium point at time $t_0$ if \[
		\forall t \geq t_0, \ f(\bs{x}_0, t) = 0 
	\]
	\label{defn:equilibrium-point}
\end{definition}
Consider a single trajectory $\phi(t, t_0, \bs{x}_0)$.
\begin{definition}
	A set $S$ is said to be the $\omega-$limit set of $\phi$ if\[
		\forall \bs{y}\in S,\exists t_n\to \infty, \lim_{n\to\infty}\phi(t_n, t_0,
		\bs{x}_0) = \bs{y}
	\]
	\label{defn:w-limit-set}
\end{definition}
Whereas linear systems converge to a single point if they converge at all,
nonlinear systems can converge to a set of points. Thus the $\omega-$limit set
essentially generalizes the idea of a limit.
\begin{definition}
	A set $M\subset \R^n$ is said to be invariant if \[
		\forall t\geq t_0,\ \bs{y}\in M \implies \phi(t, t_0, \bs{y}) \in M
	\]
	\label{defn:invariant-set}
\end{definition}
An invariant set is one which a trajectory of the system will never leave once it enters the
set. Just like linear systems, non-linear systems can also have periodic solutions.
\begin{definition}
	A closed orbit $\gamma$ is a trajectory of the system such that $\gamma(0) =
	\gamma(T)$ for finite $T$.
	\label{defn:closed-orbit}
\end{definition}

\subsection{Solutions to Nonlinear Systems}
Consider the nonlinear system \[
	\diff[]{\bs{x}}{t} = f(\bs{x}, t),\qquad \bs{x}(t_0) = \bs{x}_0\in \R^n.
\]
\begin{definition}
	A function $\bs{\Phi}(t)$ is a solution to $\diff[]{\bs{x}}{t} = f(\bs{x}, t),\
	\bs{x}(t_0) = \bs{x}_0$ on the closed interval $[t_0, t]$ if $\bs{\Phi}(t)$
	is defined on the interval $[t_0, t]$, $\diff[]{\bs{\Phi}}{t} =
	f(\bs{\Phi}(t), t)$ on the interval $[t_0, t]$, and $\bs{\Phi}(t_0) =
	\bs{x}_0$.
	\label{defn:nonlinear-solution}
\end{definition}
We say that $\bs{\Phi}(t)$ is a solution in the sense of Caratheodory if \[
	\bs{\Phi}(t) = \bs{x}_0 + \int_{t_0}^t f(\bs{\Phi}(\tau), \tau)d\tau.
\]
Because the system is nonlinear, it could potentially have no solution, one
solution, or many solutions. These solutions could exist locally, or they could
exist for all time. We might also want to know when there is a solution which
depends continuously on the initial conditions.
\begin{theorem}[Local Existence and Uniqueness]
	Given $\diff[]{\bs{x}}{t} = f(\bs{x}, t),\ \bs{x}(t_0) = \bs{x}_0\in\R^n$ where
	$f$ is piecewise continuous in $t$ and $\exists T>t_0$ such that $\forall t\in
	[t_0, T], f$ is $L$-Lipschitz Continuous, then $\exists \delta > 0$ such that a
	solution exists and is unique $\forall t\in [t_0, t_0 + \delta]$.
	\label{thm:local-existence}
\end{theorem}
\Cref{thm:local-existence} can be proved using the Contraction Mapping Theorem
(\cref{thm:contraction-mapping}) by finding $\delta$ such that the function
$P:C_n[t_0, t_0+\delta] \to C_n[t_0, t_0+\delta]$ given by \[
	P(\bs{\Phi})(t) = \bs{x}_0 + \int_{t_0}^{t_0+\delta} f(\bs{\Phi}(\tau),
	\tau)d\tau
\]
is a contraction under the norm $\|\bs{\Phi}\|_\infty = \sup_{t_0\leq t \leq
t_0+\delta} \|\bs{\Phi}(t)\|$.
\begin{theorem}[Global Existence and Uniqueness]
	Suppose $f(\bs{x}, t)$ is piecewise continuous in $t$ and $\forall T\in [t_0,
	\infty)$, $\exists L_T < \infty$ such that $f$ is $L_T$ Lipshitz continuous
	for all $\bs{x}, \bs{y} \in \R^n$, then the nonlinear system  has exactly one
	solution on $[t_0, T]$.
	\label{thm:global-existence}
\end{theorem}
Once we know that solutions to a nonlinear system exist, we can sometimes bound
them.
\begin{theorem}[Bellman-Gronwall Lemma]
	Suppose $\lambda\in\R$ is a constant and $\mu:[a,b]\to\R$ is continuous and
	non-negative, then for a continuous function $y:[a, b]\to\R$ \[
		y(t) \leq \lambda + \int_a^t \mu(\tau)y(\tau)d\tau \implies y(t) \leq
		\lambda \text{exp}\left(\int_a^t\mu(\tau)d\tau\right)
	\]
	\label{thm:bellman-gronwall}
\end{theorem}
Another thing we might want to do is understand how the nonlinear system reacts
to changes in the initial condition.
\begin{theorem}
	Suppose the system $\diff[]{\bs{x}}{t} = f(\bs{x}, t),\ \bs{x}(t_0) = \bs{x}_0$
	satisfies the conditions of global uniqueness and existence. Fix $T\in[t_0,
	\infty]$ and suppose $\bs{x}(\cdot)$ and $\bs{z}(\cdot)$  are two solutions
	satisfying $\diff[]{\bs{x}}{t} = f(\bs{x}, t), \bs{x}(t_0) = \bs{x}_0$ and
	$\diff[]{\bs{z}}{t} = f(\bs{z}(t), t),\ \bs{z}(t_0)=\bs{z}_0$, then $\forall
	\epsilon > 0, \exists \delta > 0$ such that \[
		\|\bs{x}_0 - \bs{z}_0\| < \delta \implies \|\bs{x} - \bs{z}\|_{\infty} <
		\epsilon.
	\]
	\label{thm:continuous-dependence-on-ic}
\end{theorem}
\Cref{thm:continuous-dependence-on-ic} is best understood by defining a function
$\Psi:\R^n \to C_n[t_0, t]$ where $\Psi(\bs{x}_0)(t)$ returns the solution to
the system given the initial condition. If the conditions of
\Cref{thm:continuous-dependence-on-ic} are satisfied, then the function $\Psi$
will be continuous.

\subsection{Planar Dynamical Systems}
Planar dynamical systems are those with 2 state variables.
Suppose we linearize the autonomous system $\diff[]{\bs{x}}{t} = f(\bs{x})$ at an equilibrium point.
\[
	\diff[]{\bs{x}}{t} = \frac{\partial f}{\partial \bs{x}} \bigg\lvert_{\bs{x_0}}\bs{x} 
\]
Depending on the eigenvalues of $\frac{\partial f}{\partial \bs{x}}$, the Jacobian, we get several cases for
how this linear system behaves. We'll let $z_1$ and $z_2$ be the eigenbasis of
the \textit{phase space}.
\begin{enumerate}
	\item The eigenvalues are real, yielding solutions $z_1 = z_1(0)e^{\lambda_1
		t}, z_2 = z_2(0)e^{\lambda_2 t}$. If we eliminate the time variable, we can
		plot the trajectories of the system.
		\[
			\frac{z_1}{z_1(0)} = \left(\frac{z_2}{z_2(0)}\right)^{\frac{\lambda_1}{\lambda_2}}
		\]
		\begin{enumerate}
			\item When $\lambda_1, \lambda_2 < 0$, all trajectories converge to the origin, so we call this a \textbf{stable node}.
			\item When $\lambda_1, \lambda_2 > 0$, all trajectories blow up, so we call this an \textbf{unstable node}.
			\item When $\lambda_1 < 0 < \lambda_2$, the trajectories will converge to
				the origin along the axis corresponding to $\lambda_1$ and diverge along
				the axis corresponding to $\lambda_2$, so we call this a \textbf{saddle node}.
		\end{enumerate}
	\item There is a single repeated eigenvalue with one eigenvector. As before,
		we can eliminate the time variable and plot the trajectories on the $z_1$,
		$z_2$ axes.
		\begin{enumerate}
			\item When $\lambda < 0$, the trajetories will converge to the origin, so
				we call it an \textbf{improper stable node}
			\item When $\lambda > 0$, the trajetories will diverge from the origin, so
				we call it an \textbf{improper unstable node}
		\end{enumerate}
	\item When there is a complex pair of eigenvalues, the linear system will have
		oscillatory behavior. The Real Jordan form of $\frac{\partial f}{\partial \bs{x}}$ will look like \[
			\frac{\partial f}{\partial \bs{x}} = \begin{bmatrix} \alpha & \beta \\ -\beta & \alpha \end{bmatrix}.
		\]
		The parameter $\beta$ will determine the direction of the trajectories
		(clockwise if positive).
		\begin{enumerate}
			\item When $\alpha < 0$, the trajectories will spiral towards the origin,
				so we call it a \textbf{stable focus}.
			\item When $\alpha = 0$, the trajectories will remain at a constant radius
				from the origin, so we call it a \textbf{center}.
			\item When $\alpha > 0$, the trajectories will spiral away from the
				origin, so we call it an \textbf{unstable focus}.
		\end{enumerate}
\end{enumerate}
It turns out that understanding the linear dynamics at equilibrium points can be helpful in
understanding the nonlinear dynamics near equilibrium points.
\begin{theorem}[Hartman-Grobman Theorem]
	If the linearization of a planar dynamical system $\diff[]{\bs{x}}{t} =
	f(\bs{x})$ at an equilibrium point $\bs{x_0}$ has no zero or purely imaginary eigenvalues, then there exists a
	homeomorphism from a neighborhood $U$ of $\bs{x}_0$ into $\R^2$ which takes
	trajectories of the nonlinear system and maps them onto the linearization
	where $h(\bs{x_0}) = 0$, and the homeomorphism can be chosen to preserve the
	parameterization by time.
	\label{thm:hartman-grobman}
\end{theorem}
\Cref{thm:hartman-grobman} essentially says that the linear dynamics predict the
nonlinear dynamics around equilibria, but only for a neighborhood around the
equilibrium point. Outside of this neighborhood, the linearization may be very
wrong.

Suppose that we have a simply connected region $D$ (meaning $D$ cannot be
contracted to a point) and we want to know if it contains a closed orbit.
\begin{theorem}[Bendixon's Theorem]
	If $\divergence(f)$ is not identically zero in a sub-region of $D$ and does not
	change sign in $D$, then $D$ contains no closed orbits.
	\label{thm:bendixons}
\end{theorem}
\Cref{thm:bendixons} lets us rule out closed orbits from regions of
$\R^2$. If we have a positively invariant region, then we can determine whether it
contains closed orbits.
\begin{theorem}[Poincare-Bendixson Theorem]
	If $M$ is a compact, positively invariant set for the flow $\phi_t(\bs{x})$,
	then if $M$ contains no equilibrium points, then $M$ has a limit cycle.
	\label{thm:poincare-bendixson}
\end{theorem}
\section{Stability of Nonlinear Systems}
The equilibria of a system can tell us a great deal about the stability of the
system. For nonlinear systems, stability is a property of the equilibrium
points, and to be stable is to converge to or stay equilibrium.
\begin{definition}
	An equilibrium point $\bs{x}_e\in \R$ is a stable equilibrium point in the
	sense of Lyapunov if and only if $\forall \epsilon > 0,\exists \delta(t_0,
	\epsilon)$ such that \[
		\forall t \geq t_0,\ \|\bs{x}_0 - \bs{x}_e\| < \delta(t_0, \epsilon)
		\implies \|\bs{x}(t) - \bs{x}_e\| < \epsilon
	\]
	\label{defn:stability}
\end{definition}
Lyapunov Stability essentially says that a finite deviation in the initial
condition from equilibrium means the resulting trajectory of the system stay close to
equilibrium. Notice that this definition is nearly identical to
\cref{thm:continuous-dependence-on-ic}. That means stability of an equilibrium
point is the same as saying the function which returns the solution to a system
given its initial condition is continuous at the equilibrium point.
\begin{definition}
	An equilibrium point $\bs{x}_e\in \R$ is an uniformly stable equilibrium point in the
	sense of Lyapunov if and only if $\forall \epsilon > 0,\exists \delta(
	\epsilon)$ such that \[
		\forall t \geq t_0,\ \|\bs{x}_0 - \bs{x}_e\| < \delta(\epsilon)
		\implies \|\bs{x}(t) - \bs{x}_e\| < \epsilon
	\]
	\label{defn:uniformly-stable}
\end{definition}
Uniform stability means that the $\delta$ can be chosen independently of the
time the system starts at.
Both stability and uniform stability do not imply convergence to the equilibrium
point. They only guarantee the solution stays within a particular norm ball.
Stricter notions of stabilty add this idea in.
\begin{definition}
	An equilibrium point $\bs{x}_e$ is attractive if $\forall t_0 > 0,\ \exists
	c(t_0)$ such that \[
		\bs{x}(t_0) \in B_c(\bs{x}_e) \implies \lim_{t\to\infty} \|\bs{x}(t, t_0,
		\bs{x}_0) - \bs{x}_e\| = 0
	\]
	\label{defn:attractivity}
\end{definition}
Attractive equilibria guarantee that trajectories beginning from initial
conditions inside of a ball will converge to the equilibrium. However,
attractivity does not imply stability since the trajectory could go arbitarily
far from the equilibrium so long as it eventually returns.
\begin{definition}
	An equilibrium point $\bs{x}_e$ is asymptotically stable if $\bs{x}_e$ is
	stable in the sense of Lyapunov and attractive.
	\label{defn:asymptotic-stability}
\end{definition}
Asymptotic stability fixes the problem of attractivity where trajectories could
go far from the equilibrium, and it fixes the problem with stability where the
trajectory may not converge to equilibrium. It means that trajectories starting
in a ball around equilibrium will converge to equilibrium without leaving that
ball. Because the constant for attractivity may depend on time, defining
uniform asymptotic stability requires some modifications to the idea of
attractivity.
\begin{definition}
	An equilibrium point is uniformly asympototically stable if $\bs{x}_e$ is
	uniformly stable in the sense of Lyapunov, and $\exists c$ and
	$\gamma:\R_+\times\R^n\to\R_+$ such that \[
		\forall \bs{x}_0\in B_c(\bs{x}_e),\ \lim_{\tau\to\infty}\gamma(\tau,
		\bs{x}_0) = 0, \qquad \forall t\geq t_0,\ \|\bs{x}(t, t_0, \bs{x}_0) - \bs{x}_e\| \leq
		\gamma(t-t_0, \bs{x}_0)
	\]
	\label{defn:uniform-asymptotic-stability}
\end{definition}
The existence of the $\gamma$ function helps guarantee that the rate of
converges to equilibrium does not depend on $t_0$ since the function $\gamma$ is
independent of $t_0$.
Suppose that the $\gamma$ is an exponential function. Then solutions to the
system will converge to the equilibrium exponentially fast.
\begin{definition}
	An equilibrium point $\bs{x}_e$ is locally exponentially stable if $\exists
	h,m,\alpha$
	such that \[
		\forall \bs{x}_0\in B_h(\bs{x}_e),\ \|\bs{x}(t, t_0, \bs{x}_0) -
		\bs{x}_e\| \leq me^{-\alpha(t - t_0)}\|\bs{x}(t) - \bs{x}_e\|
	\]
	\label{defn:exponential-stability}
\end{definition}
\Cref{defn:stability,defn:uniformly-stable,defn:asymptotic-stability,defn:uniform-asymptotic-stability,defn:exponential-stability}
are all local definitions because the only need to hold for $\bs{x}_0$ inside a
ball around the equilibrium. If they hold $\forall \bs{x}_0\in\R^n$, then they
become global properties.

Just as we can define stability, we can also define instability.
\begin{definition}
	An equilibrium point $\bs{x}_e$ is unstable in the sense of Lyapunov if
	$\exists \epsilon > 0, \forall \delta > 0$ such that \[
		\exists \bs{x}_0\in B_\delta(\bs{x}_e) \implies \exists T\geq t_0, x(T, t_0,
		\bs{x}_0) \not\in B_\epsilon(\bs{x}_e)
	\]
	\label{defn:unstable}
\end{definition}
Instability means that for any $\delta-$ball, we can find an $\epsilon-$ball for
which there is at least one initial condition whose corresponding trajectory
leaves the $\epsilon-$ball.
\subsection{Lyapunov Functions}
In order to prove different types of stability, we will construct functions
which have particular properties around equilibrium points of the system. The
properties of these functions will help determine what type of stable the
equilibrium point is.
\begin{definition}
	A class $\mathcal{K}$ function is a function $\alpha: \R_+ \to \R_+$ such that
	$\alpha(0) = 0$ and $\alpha(s)$ is strictly monotonically increasing in $s$.
	\label{defn:class-k-function}
\end{definition}
A subset of the class $\mathcal{K}$ functions grow unbounded as the argument
approaches infinity.
\begin{definition}
	A class $\mathcal{KR}$ function is a class $\mathcal{K}$ function $\alpha$
	where $\lim_{s\to\infty}\alpha(s) = s$.
	\label{defn:class-kr-function}
\end{definition}
Class $\mathcal{KR}$ functions are ``radially unbounded''.
We can use class $\mathcal{K}$ and class $\mathcal{KR}$ to bound ``energy-like''
functions called \textbf{Lyapunov Functions}.
\begin{definition}
	A function $V(\bs{x}, t): \R^n \times \R_+ \to \R$ is locally positive
	definite (LPDF) on a set $G\subset \R^n$ containing $\bs{x}_e$ if $\exists
	\alpha \in \mathcal{K}$ such that \[
		V(\bs{x}, t) \geq \alpha(\|\bs{x} - \bs{x}_e\|)
	\]
	\label{defn:lpdf}
\end{definition}
LPDF functions are locally ``energy-like'' in the sense that the equilibrium
point is assigned the lowest ``energy'' value, and the larger the deviation from
the equilibrium, the higher the value of the ``energy''.
\begin{definition}
	A function $V(\bs{x}, t): \R^n \times \R_+ \to \R$ is positive
	definite (PDF) if $\exists
	\alpha \in \mathcal{KR}$ such that \[
		\forall \bs{x}\in\R^n,\ V(\bs{x}, t) \geq \alpha(\|\bs{x} - \bs{x}_e\|)
	\]
	\label{defn:pdf}
\end{definition}
Positive definite functions act like ``energy functions'' everywhere in $\R^n$.
\begin{definition}
	A function $V(\bs{x}, t): \R^n \times \R_+ \to \R$ is decrescent
	if $\exists
	\alpha \in \mathcal{K}$ such that \[
		\forall \bs{x}\in B_h(\bs{x}_e),\ V(\bs{x}, t) \leq \beta(\|\bs{x} - \bs{x}_e\|)
	\]
	\label{defn:decrescent}
\end{definition}
Descresence means that for a ball around the equilibrium, we can upper bound the
the energy.

Note that we can assume $\bs{x}_e = 0$ without loss of generality for
\cref{defn:lpdf,defn:pdf,defn:decrescent} since for a given system, we can always define a linear
change of variables that shifts the equilibrium point to the origin.
\subsubsection{Quadratic Lyapunov Functions}
\begin{definition}
	A Quadratic Lypunov function is of the form \[
		V(\bs{x}) = \bs{x}^\top P \bs{x},\quad P \succ 0
	\]
	\label{defn:quadratic-lyapunov}
\end{definition}
Quadratic Lyapunov Functions are one of the simplest types of Lyapunov
Functions. Their level sets are ellipses where the major axis is the eigenvector
corresponding to $\lambda_{min}(P)$, and the minor axis is the eigenvecctor
corresponding to $\lambda_{max}(P)$.

\begin{theorem}
	Consider the sublevel set $\Omega_c = \{ \bs{x} | V(\bs{x}) \leq c \}$. Then
	$r_*$ is the radius of the largest circle contained inside $\Omega_c$, and $r^*$
	is the radius of the largest circle containing $\Omega_c$.
	\[
		r_* = \sqrt{\frac{c}{\lambda_{max}(P)}} \qquad r^* =
		\sqrt{\frac{c}{\lambda_{min}(P)}}
	\]
	\label{thm:quadratic-level-sets}
\end{theorem}

\subsubsection{Sum-of-Squares Lyapunov Functions}
\begin{definition}
	A polynomial $p(\bs{x})$ is sum-of-squares (SOS) if $\exists g_1,\cdots,g_r$
	such that \[
		p(\bs{x}) = \sum_{i=1}^r g_i^2(\bs{x})
	\]
	\label{defn:sos}
\end{definition}
SOS polynomials have the nice property that they are always non-negative due to
being a sum of squared numbers. Since any polynomial can be written in a
quadratic form $P(\bs{x}) = z^\top(\bs{x}) Q z(\bs{x})$ where $z$ is a vector of
monomials, the properties of $Q$
can tell us if $P$ is SOS or not.
\begin{theorem}
	A polynomial is SOS if and only if it can be written as \[
		p(\bs{x}) = z^\top(\bs{x}) Q z(\bs{x}), \quad Q \succeq 0
	\]
	\label{thm:sos}
\end{theorem}
Note that $Q$ is not necessarily unique, and if we construct a linear operator
which maps $Q$ to $P$, then this linear operator will have a Null Space.
Mathematically, consider \[
	\mathcal{L}(Q)(\bs{x}) = z^\top(\bs{x})Qz(\bs{x}).
\]
This linear operator has a null space spanned by the polynomials $N_j$. Given a
matrix $Q_0 \succeq 0$ such that $p(\bs{x}) = z^\top(\bs{x})Q_0z(\bs{x})$ (i.e
$p$ is SOS), it is also true that \[
	p(\bs{x}) = z^\top(\bs{x})\left( Q_0 + \sum_{j} \lambda_j N_j(\bs{x}) \right)
	z(\bs{x}).
\]
SOS polynomials are helpful in finding Lyapunov functions because we can use SOS
Programming to find SOS polynomials which satisfy desired properties. For
example, if we want $V(\bs{x})$ to be PDF, then one constraint in our SOS program
will be that \[
	V(\bs{x}) - \epsilon \bs{x}^\top \bs{x}, \quad \epsilon > 0
\] is SOS.

\subsection{Proving Stability}
To prove the stability of an equilibrium point for a given nonlinear system, we
will construct a Lyapunov function and determine stability from the properties
of the Lyapunov functions which we can find. Given properties of $V$ and $\diff[]{V}{t}$, we can use the \textbf{Lyapunov
Stability Theorems} to prove the stability of equilibria.
\begin{theorem}
	If $\exists V(\bs{x}, t)$ such that $V$ is LPDF and
	$-\diff[]{V}{t} \geq 0$ locally, then $\bs{x}_e$ is stable in the sense of
	Lyapunov.
	\label{thm:lyapunov-sisl}
\end{theorem}
\begin{theorem}
	If $\exists V(\bs{x}, t)$ such that $V$ is LPDF and decrescent, and 
	$-\diff[]{V}{t} \geq 0$ locally, then $\bs{x}_e$ is uniformly stable in the sense of
	Lyapunov.
	\label{thm:lyapunov-uniformly-sisl}
\end{theorem}
\begin{theorem}
	If $\exists V(\bs{x}, t)$ such that $V$ is LPDF and decrescent, and 
	$-\diff[]{V}{t}$ is LPDF, then $\bs{x}_e$ is uniformly asymptotically stable in the sense of
	Lyapunov.
	\label{thm:lyapunov-uniformly-as}
\end{theorem}
\begin{theorem}
	If $\exists V(\bs{x}, t)$ such that $V$ is PDF and decrescent, and 
	$-\diff[]{V}{t}$ is LPDF, then $\bs{x}_e$ is globally uniformly asymptotically stable in the sense of
	Lyapunov.
	\label{thm:lyapunov-globally-as}
\end{theorem}
\begin{theorem}
	If $\exists V(\bs{x}, t)$ and $h, \alpha > 0$ such that V is LPDF is
	decrescent, $-\diff[]{V}{t}$ is LDPF, and \[
		\forall \bs{x}\in B_h(\bs{x}_e),\ \left\lvert\left\lvert\diff[]{V}{t}\right\rvert\right\rvert \leq \alpha \|\bs{x}-\bs{x}_e\|
	\]
	\label{thm:lyapunov-exponential-stability}
\end{theorem}
The results of
\cref{thm:lyapunov-sisl,thm:lyapunov-uniformly-sisl,thm:lyapunov-uniformly-as,thm:lyapunov-globally-as,thm:lyapunov-exponential-stability}
are summarized in \cref{table:lyapunov-stability}.
\begin{gitbook-image}
\begin{table}[!h]
	\centering
	\begin{tabularx}{\textwidth}{YYY}
		\toprule
		\textbf{Conditions on} $\mathbf{V}$ & \textbf{Conditions on}
		$-\diff[]{V}{t}$ & \textbf{Conclusion} \\
		\midrule
		LPDF & $\geq 0$ locally & Stable\\
		\hline
		LPDF, Decrescent & $\geq 0$ locally & Uniformly Stable\\
		\hline
		LPDF, Decrescent & LPDF & Uniformly, Asymptotically Stable\\
		\hline
		LPDF, Decrescent & LDPF, $\exists \alpha > 0$ such that $\left\lvert\left\lvert
		\diff[]{V}{t}\right\rvert\right\rvert \leq \alpha \|\bs{x} - \bs{x}_e\|$ &
		Exponentially Stable\\
		\hline
		PDF, Decrescent & PDF & Globally, Uniformly, Asymptotically Stable\\
		\bottomrule
	\end{tabularx}
	\caption{Summary of Lyapunov Stability Theorems}
	\label{table:lyapunov-stability}
\end{table}
\end{gitbook-image}
Going down the rows of \cref{table:lyapunov-stability} lead to increasingly
stricter forms of stability. Descresence appears to add uniformity to the
stability, while $-\diff[]{V}{t}$ being LPDF adds asymptotic convergence. However,
these conditions are only sufficient, meaning if we cannot find a suitable $V$,
that does not mean that an equilibrium point is not stable.

One very common case where it can be difficult to find appropriate Lyapunov
functions is in proving asymptotic stability since it can be hard to find $V$
such that $-\diff[]{V}{t}$ is LPDF. In the case of autonomous systems, we can
still prove asymptotic stability without such a $V$.
\begin{theorem}[LaSalle's Invariance Principle]
	Consider a smooth function $V:\R^n\to\R$ with bounded sub-level sets $\Omega_c
	= \left\{\bs{x} | V(\bs{x}) \leq c \right\}$ and $\forall \bs{x}\in \Omega_c$, 
	$\diff[]{V}{t} \leq 0$. Define $S = \left\{\bs{x}\bigg\lvert\diff[]{V}{t} = 0\right\}$ and let
		$M$ be the largest invariant set in $S$, then \[
			\forall \bs{x}_0\in \Omega_c,\ \bs{x}(t, t_0, x_0) \to M \text{ as } t\to
			\infty.
		\]
	\label{thm:la-salles}
\end{theorem}
LaSalle's theorem helps prove general convergence to an invariant set. Since $V$
is always decreasing in the sub-level set $\Omega_c$, trajectories starting in
$\Omega_c$ must eventually reach $S$. At some point, they will reach the set $M$
in $S$, and then they will stay there. Thus if the set $M$ is only the
equilibrium point, or a set of equilibrium points, then we can show that the
system trajectories asymptotically converges to this equilibrium or set of
equilibria. Moreover, if $V(\bs{x})$ is PDF, and $\forall \bs{x}\in\mathbb{R}^n,
\diff[]{V}{t} \leq 0$, then we can show global asymptotic stability as well.

LaSalle's theorem can be generalized to non-autonomous systems as well, but it
is slightly more complicated since the set $S$ may change over time.

\subsubsection{Indirect Method of Lyapunov}
It turns out that we can also prove the stability of systems by looking at the
linearization around the equilibrium. Without loss of generality, suppose
$\bs{x}_e = 0$. The linearization at the equilibrium is given by \[
	\diff[]{\bs{x}}{t} = f(x,t) = f(0, t) + \frac{\partial f}{\partial
	\bs{x}}\bigg\lvert_{\bs{x} = 0}\bs{x} + f_1(\bs{x},t) \approx A(t)\bs{x}.
\]
The function $f_1(\bs{x}, t)$ is the higher-order terms of the linearization.
The linearization is a time-varying system.
Consider the time-varying linear system
\[
	\diff[]{\bs{x}}{t} = A(t)\bs{x},\ \bs{x}(t_0) = \bs{x}_0.
\]
\begin{definition}
	The state transition matrix $\Phi(t, t_0)$ of a time-varying linear system is
	a matrix satisfying \[
		\bs{x}(t) = \Phi(t, t_0)\bs{x}_0,\ \diff[]{\Phi}{t} = A(t)\Phi(t, t_0),\
		\Phi(t_0, t_0) = I
	\]
	\label{defn:state-transition-matrix}
\end{definition}
The state transition matrix is useful in determining properties of the system.
\begin{enumerate}
	\item $\sup_{t\geq t_0} \|\Phi(t, t_0)\| = m(t_0) < \infty \implies$ the system
		is stable at the origin at $t_0$.
	\item $\sup_{t_0\geq 0}\sup_{t\geq t_0} \|\Phi(t, t_0)\| = m < \infty \implies$ the system
		is uniformly stable at the origin at $t_0$.
	\item $\lim_{t\to\infty}\|\Phi(t, t_0)\| = 0 \implies$ the system is
		asymptotically stable.
	\item $\forall t_0,\epsilon>0,\exists T$ such that $\forall t\geq t_0 + T,\ \|\Phi(t, t_0)\| <
		\epsilon \implies$ the system is uniformly asymptotically stable.
	\item $\|\Phi(t, t_0)\| \leq Me^{-\lambda(t-t_0)} \implies$ exponential
		stability.
\end{enumerate}
If the system was Time-Invariant, then the system would be stable so long as the
eigenvalues of $A$ were in the open left-half of the complex plane. In fact, we
could use $A$ to construct positive definite matrices.
\begin{theorem}[Lyapunov Lemma]
	For a matrix $A\in \R^{n\times n}$, its eigenvalues $\lambda_i$ satisfy
	$\Re(\lambda_i) < 0$ if and only if $\forall Q \succ 0$, there exists
	a solution $P\succ 0$ to the equation \[
		A^TP + PA = -Q.
	\]
	\label{thm:lyapunov-lemma}
\end{theorem}
In general, we can use the \textbf{Lyapunov Equation} to count how many
eigenvalues of $A$ are stable.
\begin{theorem}[Tausskey Lemma]
	For $A\in\R^{n\times n}$ and given $Q \succ 0$, if there are no eigenvalues on
	the $j\omega$ axis, then the solution $P$ to $A^TP + PA = -Q$ has as many
	positive eigenvalues as $A$ has eigenvalues in the complex left half plane.
	\label{thm:tausskey-lemma}
\end{theorem}
The Lyapunov Lemma has extensions to the time-varying case.
\begin{theorem}[Time-Varying Lyapunov Lemma]
	If $A(\cdot)$ is bounded and for some $Q(t) \succeq \alpha I$, the solution
	$P(t)$ to $A(t)^TP(t) + P(t)A(t) = -Q(t)$ is bounded, then the origin is a
	asymptotically stable equilibrium point.
	\label{thm:tv-lyapuynov-lemma}
\end{theorem}
It turns out that uniform asymptotic stability of the linearization of a system
corresponds to uniform, asymptotic stability of the nonlinear system.
\begin{theorem}[Indirect Theorem of Lyapunov]
	For a nonlinear system whose higher-order terms of the linearization are given
	by $f(\bs{x},t)$, if \[
		\lim_{\|\bs{x}\|\to 0}\sup_{t\geq 0} \frac{\|f_1(\bs{x},t)\|}{\|\bs{x}\|} =
		0
	\]
	and if $\bs{x}_e$ is a uniformly asymptotic stable equilibrium point of
	$\diff[]{\bs{z}}{t}=A(t)\bs{z}$ where $A(t)$ is the Jacobian at the $\bs{x}_e$,
	then $\bs{x}_e$ is a uniformly asymptotic stable
	equilibrium point of $\diff[]{\bs{x}}{t} = f(\bs{x},t)$
	\label{thm:indirect-lyapunov}
\end{theorem}
\subsection{Proving Instability}
\begin{theorem}
	An equilibrium point $\bs{x}_e$ is unstable in the sense of Lyapunov if
	$\exists V(\bs{x},t)$ which is decrescent, the Lie derivative $\diff[]{V}{t}$ is
	LPDF, $V(\bs{x}_e, t)$, and $\exists \bs{x}$ in the neighborhood of $\bs{x}_e$
	such that $V(\bs{x}_0, t) > 0$.
	\label{thm:lyapunov-instability}
\end{theorem}
\subsection{Region of Attraction}
For asymptotically stable and exponential stable equilibria, it makes sense to
wonder which initial conditions will cause trajectories to converge to the
equilibrium.
\begin{definition}
	If $\bs{x}_e$ is an equilibrium point of a time-invariant system $\diff[]{\bs{x}}{t}
	= f(\bs{x})$, then the Region of Attraction of $\bs{x}_e$ is \[
		\mathcal{R}_A(\bs{x}_e) = \{ \bs{x}_0 \in \mathbb{R}^n | \lim_{t\to\infty}
		\bs{x}(t, t_0) = \bs{x}_e \}
	\]
	\label{defn:region-of-attraction}
\end{definition}
Suppose that we have a Lyapunov function $V(\bs{x})$ and a region $D$ such that
$V(\bs{x}) > 0$ and $\diff[]{V}{t} < 0$ in $D$. Define a sublevel set of the
Lypunov function $\Omega_c$ which is a subset of $D$. We know that if
$\bs{x}_0\in\Omega_c$, then the trajectory will stay inside $\Omega_c$ and
converge to the equilibrium point. Thus we can use the largest $\Omega_c$ that
is compact and contained in $D$ as an estimate of the region of attraction.

When we have a Quadratic Lyapunov Function, we can set $D$ to be the largest
circle which satisfies the conditions on $V$, and the corresponding $\Omega_c$
contained inside $D$ will be the estimate of the Region of Attraction.

We can find even better approximations of the region of attraction using SOS
programming. Suppose we have a $V$ which we used to prove asymptotic stability.
Then if there exists an $s$ which satisfies the following SOS program, then the
sublevel set $\Omega_c$ is an estimate of the Region of Attraction.
\begin{align*}
	\max_{c, s} &\quad c\\
	\text{s.t} &\quad s(\bs{x}) \text{ is SOS,}\\
	&\quad -\left(\diff[]{V}{t} + \epsilon \bs{x}^\top \bs{x}\right) + s(\bs{x})(c -
	V(\bs{x})) \text{ is SOS.}
\end{align*}
\section{Nonlinear Feedback Control}
In nonlinear control problems, we have a system \[
	\diff[]{\bs{x}}{t} = f(\bs{x}, \bs{u}).
\]
$\bs{x}$ is the state of the system, and $\bs{u}$ is the input to the system.
Note that for simplicity, the system is time-invariant. Further assume, without
loss of generality, that $f(0, 0) = 0$. The goal of nonlinear feedback control
is to find a state feedback law $\alpha(\bs{x})$ such that the equilibrium point
$\bs{x}_e = 0$ is globally asymptotically stable for the closed loop system \[
	\diff[]{\bs{x}}{t} = f(\bs{x}, \alpha(\bs{x})).
\]
Sometimes, the control impacts the state evolution in an affine manner.
\begin{definition}
	A control affine system is given by the differential equation \[
		\diff[]{\bs{x}}{t} = f(\bs{x}) + G(\bs{x})\bs{u}
	\] where $G(\bs{x})$ is a matrix dependent on the state vector $\bs{x}$.
	\label{defn:control-affine}
\end{definition}

When designing controllers, there is a wide variety of techniques we can use.
Some simple techniques involve canceling out various types of nonlinearities in
the system using the input. Here are some examples.
\begin{enumerate}
	\item Set $\bs{u}$ such that it cancels out nonlinear terms and adds a stable
		linear term, effectively making the nonlinear system behave linear in the
		closed loop.
	\item Set $\bs{u}$ to cancel destabilitizing nonlinear terms and add a stable
		linear term, so the stable nonlinearities help the input drive the system to
		equilibrium.
	\item Set $\bs{u}$ to cancel destabilizing nonlinear terms, so the nonlinear
		system dynamics drive the system to equilibrium.
	\item Set $\bs{u}$ to dominate destabilizing terms so they have a minimal
		impact on the overall system behavior.
\end{enumerate}
While these techniques can work, there are also more principled ways of
designing controllers to satisfy different criteria, particularly for the case
of control affine systems.
\subsection{Control Lyapunov Functions}
If we can find an $\alpha(\bs{x})$ that makes the origin globally asymptotically
stable, then the converse Lyapunov theorem says that we can find a
corresponding Lyapunov function for the system.
\begin{align*}
	\forall \bs{x} \neq 0, \diff[]{V}{t} < 0 &\implies (\nabla_{\bs{x}}V)^\top f(\bs{x}, \alpha(\bs{x})) < 0 \\
	&\implies \exists \bs{u} \text{ s.t } (\nabla_{\bs{x}}V)^\top f(\bs{x},
	\bs{u}) < 0\\
	& \Leftrightarrow \inf_{\bs{u}} (\nabla_{\bs{x}}V)^\top f(\bs{x}, \bs{u}) < 0
\end{align*}
This result motivates the following definition.
\begin{definition}
	A continuously differentiable, PDF, radially unbounded $V: \R^n \to \R$ is a
	Control Lyapunov Function for the system $\diff[]{\bs{x}}{t} = f(\bs{x},
	\bs{u})$ if \[
		\forall \bs{x} \neq 0, \inf_{\bs{u}} (\nabla_{\bs{x}}V)^\top
		f(\bs{x}, \bs{u}) < 0
	\]
	\label{defn:control-lyapunov-function}
\end{definition}
Once we have a control lyapunov function, we can prove that it is possible to
find a state feedback law that will make the origin globally asymptotically
stable.
\begin{theorem}
	Suppose $f$ is Lipschitz and $V$ is a control Lyapunov function, then there
	exists a smooth function $\alpha$ such that the origin is a globally
	asympototically stable equilibrium point of $\diff[]{\bs{x}}{t} = f(\bs{x},
	\alpha(\bs{x}))$.
	\label{thm:artstein}
\end{theorem}
Suppose that we have a control affine system, and we want to construct a control
lyapunov function for the system. \[
	\inf_{\bs{u}} \frac{\partial V}{\partial \bs{x}}f(\bs{x}, \bs{u}) =
	\inf_{\bs{u}} L_f V + \sum_i L_{g_i} V u_i < 0
\]
Here, each $g_i(\bs{x})$ is a column of $G(\bs{x})$.
If $\forall i,\ L_{g_i} V = 0$, then \cref{defn:control-lyapunov-function} is
satisfied so long as $L_f V < 0$.
\begin{theorem}
	A function $V$ is a control lyapunov function for a control affine system \[
		\diff[]{\bs{x}}{t} = f(\bs{x}) + \sum_i g_i(\bs{x}) u_i
	\] if \[
		L_{g_i} V\, \forall i \implies L_f V \leq 0
	\]
	\label{thm:control-affine-lyap}
\end{theorem}
Notice that the condition in \cref{thm:control-affine-lyap} is essentially
saying that the $l_2$-norm of the vector composed of the $L_{g_i} V$ is equal to
0. The choice of CLF is important because different CLFs have different
properties when used to derive controllers.
\begin{definition}
	A CLF $V(\bs{x})$ satisfies the small control property if $\forall \epsilon >
	0$, $\exists \delta > 0$ such that $\bs{x}\in B_\delta(0)$, then if
	$\bs{x}\neq0, \exists \bs{u}\in B_\epsilon(0)$ satisfying \[
		\diff[]{V}{t} = L_fV + L_GV^T\bs{u} < 0.
	\]
	\label{defn:small-control}
\end{definition}
The small control property means that CLF will lead to a controller which has a
small value that does not get too large when close to the equilibrium.

Given a control lyapunov function for a control affine system $V(\bs{x},
\bs{u})$, we can devise
a controller which stabilizes the system. In particular, we need \[
	\frac{dV}{dt}(x, u) = L_f V(\bs{x}) + L_GV(\bs{x})^\top \bs{u} \leq 0.
\]
Hence, let \[
	\bs{u} = \begin{cases}
		0, & \text{if } L_f V < 0,\\
		(L_GVL_GV^\top)^{-1}(-L_fV L_GV^\top), & \text{if } L_f V > 0.
	\end{cases}
\]
When the plant dynamics are naturally stabilizing, this controller exerts no
control effort. When the plant dynamics are not naturally stabilizing, then the
controller applies some control to stabilize the system. We can show that this
is a minimum norm controller as it solves the optimization problem \[
	\begin{aligned}
		\min &\quad \bs{u}^\top \bs{u}\\
		\text{s.t} &\quad L_fV + L_GV^\top \bs{u} \leq 0.
	\end{aligned}
\]
Another type of controller is known as the Sontag controller.
\begin{theorem}
	Suppose $V:\R^n\to\R$ is a CLF for SISO control affine system $\diff[]{\bs{x}}{t} =
	f(\bs{x}) + g(\bs{x})u$ where $f(0) = 0$ where $f$ and $g$ are Lipschitz. Then
	the Sontag feedback control law is given by \[
		\alpha_S(\bs{x}) = \begin{cases}
			\frac{-L_fV - \sqrt{(L_fV)^2 + (L_gV)^4}}{L_gV}, & \text{if } L_gV \neq
			0,\\
			0, & \text{else.}
		\end{cases}
	\]
	makes the origin globally asymptotically stable. Moreover,
	$\alpha_S(\bs{x})$ is continuous everywhere except $\bs{x} = 0$, is
	continuous at $\bs{x}=0$ if $V$ satisfies the small control property, and if
	$V(\bs{x})$ is $K+1$ times continuously differentiable and $f(\bs{x}),
	g(\bs{x})$ are $K$ times continously differentiable $\forall \bs{x}\neq0$,
	then $\alpha_S$ is $K$ times continuously differentiable.
	\label{thm:sontag}
\end{theorem}
\subsection{Feedback Linearization}
Since we have a large number of tools which allow us to control linear systems,
it would be ideal if we could somehow leverage those tools for nonlinear
control. Feedback linearization is the process of finding a feedback control law $\bs{u} =
\alpha(\bs{x})$ such that under a nonlinear change of coordinates $z =
\Phi(\bs{x})$, the system $\diff[]{\bs{x}}{t} = f(\bs{x}, \alpha(\bs{x}))$ behaves like a linear
system $\diff[]{\bs{z}}{t} = A\bs{z}$. When the system is control-affine, there
are well-established results which help us do this.

\subsubsection{SISO Case}
Suppose we have a SISO control-affine system \[
	\begin{aligned}
		\diff[]{\bs{x}}{t} &= f(\bs{x}) + g(\bs{x})u\\
		y &= h(\bs{x})
	\end{aligned}
\]
\begin{definition}
	A SISO control affine system with an equilibrium point $\bs{x}_e$ has
	strict relative degree $\gamma$ if in a neighborhood $U$ around the
	equilibirum point, $L_gL_f^{\gamma-1}h(\bs{x})$ is bounded away from 0 and \[
		\forall i=0,\cdots,\gamma-2,\ L_gL_f^{\gamma - 1}h(\bs{x}) = 0
	\]
	\label{defn:strict-relative-degree}
\end{definition}
To understand relative degree, suppose we differentiate $y$ once \[
	\diff[]{y}{t} = L_fh + L_gh u.
\]
If $\forall \bs{x}\in U,\ L_gh(\bs{x}) = 0$ where $U$ is some region around the
equilibrium, then \[
	\forall \bs{x}\in U,\ \diff[]{y}{t} = L_fh(\bs{x}).
\]
If we differentiate again, then \[
	\forall \bs{x}\in U,\ \diff[2]{y}{t} = L_f^2h(\bs{x}) + L_gL_fh(\bs{x}).
\]
Suppose that $\forall \bs{x}\in U,\ L_gL_fh(\bs{x}) = 0$, then we can
differentiate again. At some point, after $\gamma$ differentiations, we will
get \[
	\forall \bs{x}\in U,\ \diff[\gamma]{y}{t} = L_f^\gamma h(\bs{x}) +
	L_gL_f^{\gamma-1}h(\bs{x})u.
\]
Therefore, the relative degree of the system is essentially telling us which
derivative of the output that we can control. By sequentially taking
derivatives, we are essentially looking at the system \[
	\begin{aligned}
		y &= h(\bs{x})\\
		\diff[]{y}{t} &= L_fh(\bs{x})\\
		\diff[2]{y}{t} &= L_f^2h(\bs{x})\\
		&\vdots\\
		\diff[\gamma]{y}{t} &= L_f^\gamma h(\bs{x}) + L_gL_f^{\gamma-1}h(\bs{x})
	\end{aligned}
\]
Suppose $\forall i=0,\cdots,\gamma-1,$ we let $\xi_i(\bs{x}) = \diff[i]{y}{t}$.
These are $\gamma$ linearly independent coordinates. Since the distribution \[
	\Delta(\bs{x}) = \text{span}\{g(\bs{x})\}
\]
is involutive, it is integrable, and so there must be $n-1$ functions $\eta_i$
such that \[
	\forall \bs{x}\in U,\ (\nabla_{\bs{x}}\eta_i)^\top g(\bs{x}) = 0.
\]
We can now choose $n-\gamma$ of them which are linearly independent of the
$\xi_i$ and linearly independent with each other, and this forms a change of
coordinates \[
	\begin{bmatrix}
		\xi_1\\
		\vdots\\
		\xi_{\gamma}\\
		\eta_1\\
		\vdots\\
		\eta_{n-\gamma}
	\end{bmatrix} = \Phi(\bs{x}) = \begin{bmatrix}
		h(\bs{x})\\
		L_fh(\bs{x})\\
		\vdots\\
		L_f^{\gamma-1}\\
		\eta_1\\
		\vdots\\
		\eta_{n-\gamma}
	\end{bmatrix}.
\]
This change of coordinates allows us to put the system into a canonical form.
\begin{definition}
	The normal form of a SISO control affine system is given by \[
		\begin{aligned}
			\diff[]{\xi_1}{t} &= \xi_2\\
			\diff[]{\xi_2}{t} &= \xi_3\\
			&\vdots\\
			\diff[]{\xi_{\gamma}}{t} &= b(\bs{\xi}, \bs{\eta}) + a(\bs{\eta},
			\bs{\xi})u\\
			\diff[]{\bs{\eta}}{t} &= q(\bs{\xi}, \bs{\eta}),\\
			y &= \eta_1
		\end{aligned}
	\]
	\label{defn:normal-form-siso}
\end{definition}
When the original system is given by $\diff[]{\bs{x}}{t} = f(\bs{x})+g(\bs{x})u$, then
\[
	b(\bs{\xi},\bs{\eta}) = L_f^\gamma h(\Phi^{-1}(\bs{\xi}, \bs{\eta})) \qquad
	a(\bs{\xi}, \bs{\eta}) = L_gL_f^{\gamma-1} h(\Phi^{-1}(\bs{\xi}, \bs{\eta}))
\]
With this parameterization, it is quite easy to see how we can make our system
behave linearly. In particular, choose \[
	u = \frac{1}{a(\bs{\xi}, \bs{\eta})}\left(-b(\bs{\xi}, \bs{\eta}) +
	v\right)
\]
where $v$ is some control input. Then the system becomes
\[
	\begin{aligned}
		\diff[]{\xi_1}{t} &= \xi_2\\
		\diff[]{\xi_2}{t} &= \xi_3\\
		&\vdots\\
		\diff[]{\xi_{\gamma}}{t} &= v\\
		\diff[]{\bs{\eta}}{t} &= q(\bs{\xi}, \bs{\eta})\\
		y &= \eta_1
	\end{aligned}
\],
which is a linear system. Therefore, we can design a linear feedback controller $v
= \alpha(\bs{x})$
where we have all of the tools of linear control at our disposal. However,
notice that the $\eta_i$ cannot be impacted by the control effort. These are
known as the \textbf{internal dynamics} of the system. When $\bs{\xi} = 0$,
then \[
	\diff[]{\bs{\eta}}{t} = q(0, \bs{\eta})
\]
are known as the \textbf{Zero Dynamics} of the system. Zero dynamics for a
system can be dangerous because if they are unstable, then the system could be
blowing up. When $\gamma = n$, then there are no zero dynamics. When this
happens, we say the system is \textbf{Full State Linearizable}. Fortunately,
there are necessary and sufficient conditions which guarantee full state
linearization.
\begin{theorem}
	There exists a function $h$ such that a control affine system
	$\diff[]{\bs{x}}{t} = f(\bs{x}) + g(\bs{x})u$ has relative degree $n$ at
	$\bs{x}_0$ if and only if \[
		\begin{bmatrix}
			g(\bs{x}) & \cdots & \ad_f^{n-2}g(\bs{x}) & \ad_f^{n-1}g(\bs{x})
		\end{bmatrix}
	\]
	has rank $n$ and \[
		\begin{bmatrix}
			g(\bs{x}) & \cdots & \ad_f^{n-3}g(\bs{x}) & \ad_f^{n-2}g(\bs{x})
		\end{bmatrix}
	\]
	has rank $n-1$ and is involutive in the neighborhood of $\bs{x}_0$. The $h$ is
	chosen to satisfy \[
		(\nabla_{\bs{x}}h)^\top \begin{bmatrix}
			g(\bs{x}) & \cdots & \ad_f^{n-3}g(\bs{x}) & \ad_f^{n-2}g(\bs{x})
		\end{bmatrix} = 0
	\]
	\label{thm:full-state-linearization}
\end{theorem}
\subsubsection{MIMO Case}
Suppose instead we have a MIMO control affine system where \[
	\begin{aligned}
		\diff[]{\bs{x}}{t} &= f(\bs{x}) + G(\bs{x})\bs{u}\\
		\bs{y} &= h(\bs{x})
	\end{aligned}
\]
We will assume that the number of outputs is equal to the number of inputs (i.e
$\bs{y}, \bs{u} \in \R^m$). To linearize the system, we can take the same idea
of relative degree from the SISO case and apply it to the MIMO case.
Define $\gamma_j$ to be the lowest derivative of $y_j$ which is impacted by at
least one input.\[
	\begin{bmatrix}
		\diff[\gamma_1]{y_1}{t}\\
		\vdots\\
		\diff[\gamma_m]{y_m}{t}
	\end{bmatrix}
	= \begin{bmatrix}
		L_f^{\gamma_1}h_1(\bs{x})\\
		\vdots\\
		L_f^{\gamma_m}h_m(\bs{x})\\
	\end{bmatrix} + A(\bs{x})\bs{u}, \qquad A(\bs{x}) = \begin{bmatrix}
		L_{g_1}L_f^{\gamma_1-1}h_1(\bs{x}) & \cdots &
		L_{g_m}L_f^{\gamma_1-1}h_1(\bs{x})\\
		\vdots & \ddots & \vdots\\
		L_{g_1}L_f^{\gamma_m-1}h_m(\bs{x}) & \cdots &
		L_{g_m}L_f^{\gamma_m-1}h_m(\bs{x})
	\end{bmatrix}
\]
\begin{definition}
	A square control affine system has a vector relative degree $(\gamma_1,
	\cdots, \gamma_m)$ at $\bs{x}_0\in U$ if $A(\bs{x}_0)$ is nonsingular and \[
		\forall 1\leq i\leq m,\ 1\leq j \leq m,\ 0 \leq k \leq \gamma_j-2,\ \forall
		\bs{x}\in U,\ L_{g_i}L_f^kh_j(\bs{x}) = 0
	\]
	\label{thm:vector-relative-degree}
\end{definition}
As before, we can assign $\diff[i]{y_j}{t} = \xi_i^j$ as a partial change of
coordinates and then choose linearly independent $\bs{\eta}$.
\begin{definition}
	The normal form of a square MIMO system is given by
	\[
		\begin{aligned}
			\diff[]{\bs{\eta}}{t} &= q(\bs{\xi},\bs{\eta}) + p(\bs{\xi}, \bs{\eta})\bs{u}\\
			\diff[]{\xi_i^j}{t} &= \xi_{i+1}^j, &\quad \forall j,\ \forall i < \gamma_j-1\\
			\diff[]{\xi_{\gamma_j-1}^j}{t} &= b^j(\bs{\xi},\bs{\eta}) +
			\bs{a}^j(\bs{\xi},\bs{\eta})^\top\bs{u}
		\end{aligned}
	\]
	\label{defn:mimo-normal-form}
\end{definition}
As before the $\diff[]{\bs{\eta}}{t}$ represent the internal dynamics of the
system that are not impacted by the control. As with the linear case, we can
design a controller \[
	\bs{u} = A^{-1}(\bs{x})\left(
	\begin{bmatrix}
		L_f^{\gamma_1}h_1(\bs{x})\\
		\vdots\\
		L_f^{\gamma_m}h_m(\bs{x})\\
	\end{bmatrix} + \bs{v}\right)
\]
which renders the system linear. We can now choose $\bs{v}$ where each entry of
$\bs{v}$ controls a different output. For this reason, we call $A(\bs{x})$ the
decoupling matrix. As in the SISO case, unless $\sum_j \gamma_j = n$, there are
zero dynamics to the system.
\begin{theorem}
	A control affine square system $\diff[]{\bs{x}}{t} = f(\bs{x}) +
	G(\bs{x})\bs{u}$ has vector relative degree $\sum_j \gamma_j = n$ if and only
	if $\Delta_i$ is involutive for all $i \leq n-2$, $\Delta_i$ has constant rank
	for all $1 \leq i \leq n-1$ and $\Delta_{n-1}$ has rank $n$ where \[
		\begin{aligned}
			\Delta_0(\bs{x}) &= \text{span}\{g_1(\bs{x}),\cdots,g_m(\bs{x})\}\\
			\Delta_i(\bs{x}) &= \text{span}\{\ad_f^kg_i(\bs{x})\ |\  \forall 0 \leq k \leq i,
			1 \leq j \leq m\}, &\quad \forall 1 \leq i \leq n-1
		\end{aligned}
	\]
	\label{thm:full-state-linearization-mimo}
\end{theorem}
\subsubsection{Dynamic Extension}
Sometimes, we can use full-state linearization even if $h$ does not satisfy the
conditions in \cref{thm:full-state-linearization}. We do this by adding
additional states to the system and corresponding pseudo-control inputs which
help control these states. Sometimes, this can be done in a way which makes the
extended system full-state linearizable.
\subsubsection{Sliding Mode Control}
In sliding mode control, we design a controller \[
	u = \beta(\bs{x})\text{sgn}(s)
\]
where $s(\bs{x})$ describes a manifold called the ``sliding manifold''. Sliding mode
controllers have two states
\begin{enumerate}
	\item Reaching Mode
	\item Sliding Mode
\end{enumerate}
During the reaching mode, the controller drives the state towards the sliding
manifold $s(\bs{x}) = 0$. We choose $s$ such that on the manifold, when the
system is in sliding mode, the system
naturally converges asymptotically to the equilibrium. If $s(\bs{x}) = 0$ is an
invariant manifold, then the system will smoothly travel along the manifold to
equilibrium. If the sliding manifold is not invariant, then the state will
chatter around the manifold towards equilibrium as the controller continuously
drives it back to the manifold once it leaves. To choose $s$, we need to find a CLF $V(s)$ which
converges to $0$ in finite time when applying the sliding mode controller.
\subsubsection{Backstepping}
\begin{definition}
	A system expressed in strict feedback form is given by \[
		\begin{aligned}
			\diff[]{\bs{x}}{t} &= f_0(\bs{x}) + g_0(\bs{x})\xi_1\\
			\diff[]{\xi_1}{t} &= f_1(\bs{x}, \xi_1) + g_1(\bs{x}, \xi_1)\xi_2\\
			&\vdots\\
			\diff[]{\xi_k}{t} &= f_k(\bs{x}, \xi_1,\cdots,\xi_k) + g_k(\bs{x},
			\xi_1,\cdots,\xi_k)u\\
		\end{aligned}
	\]
	\label{defn:strict-feedback-form}
\end{definition}
When systems are expressed in this way, we have a convenient method of designing
controllers.
\begin{theorem}[Backstepping Lemma]
	Suppose there is a continuously differentiable $u=\alpha(\bs{x})$ and a CLF
	$V(\bs{x})$ such that \[
		L_fV + L_gV\alpha \leq -W
	\] where $W$ is a positive semi-definite function for the system
	$\diff[]{\bs{x}}{t} = f(\bs{x})+g(\bs{x})u$. Then for the system \[
		\begin{aligned}
			\diff[]{\bs{x}}{t} &= f(\bs{x}) + g(\bs{x})\xi\\
			\diff[]{\xi}{t} &= u
		\end{aligned}
	\]
	the function \[
		V_a(\bs{x},\xi) = V(\bs{x}) + \frac{1}{2}(\xi - \alpha(\bs{x}))^2
	\] is a valid CLF and the control input \[
		u = -c(\xi-\alpha(\bs{x})) + (\nabla_{\bs{x}}\alpha)^\top(f(\bs{x}) +
		g(\bs{x})\xi) - (\nabla_{\bs{x}}V)^\top g(\bs{x}), \qquad c > 0
	\]
	is a stabilizing controller.
	\label{thm:backstepping}
\end{theorem}
If we apply \cref{thm:backstepping} to a system expressed in strict feedback
form, then we can recursively define controllers until we arrive at a controller
for the full system.
\end{document}

