\documentclass[12pt]{article}

\setlength{\topmargin} {-1in}

% Style modifications
\oddsidemargin  0.25in
\evensidemargin 0.25in
\textwidth      6.0in
%\textheight     8.0in
%\textheight     8.5in % Changed on 13 September 2012
\textheight     9in % Changed on 3 October 2013
\parskip        0.1in
\parindent      0.0in
\headheight     1.0in

\headsep        .25in

\usepackage{comment} % For gitbook support

\usepackage{graphicx}
\usepackage{amsmath,amsfonts,amssymb,amscd,verbatim,graphicx,fancyhdr,tikz}
\usepackage{mathtools} %for minus plus sign alignment in matrices and vectors, added 13 Feb 2017
\usepackage{bbm}
\usepackage[ruled,vlined]{algorithm2e}

\usepackage{pgfplots}
\usetikzlibrary{arrows,automata}
\usetikzlibrary{positioning}
\usetikzlibrary{shapes.geometric}

\usepgfplotslibrary{groupplots}
\pgfplotsset{compat=1.16}

\usepackage{tikz} 
\usetikzlibrary{arrows.meta}
\usepackage{circuitikz}

\usepackage{esdiff}
\usepackage{siunitx}

\usepackage{palatino}
\usepackage{enumerate}
\usepackage{multicol}
\usepackage{listings}
\usepackage{color}
\usepackage{float}
\usepackage{esdiff} % Derivatives
\usepackage{hyperref}
\usepackage{cleveref}
\usepackage{bookmark}
\usepackage{slashbox}
\usepackage{tabularx}
\usepackage{booktabs}
\usepackage{siunitx}

\newcolumntype{Y}{>{\centering\arraybackslash}X}

\hypersetup{
    colorlinks=true,
    linkcolor=blue,
    filecolor=magenta,      
    urlcolor=blue,
}

\newcommand{\bblue}{\color[rgb]{0.2,0.2,0.7}}

\usepackage{mcode}
%Alternative: http://perso.telecom-bretagne.eu/gillesbertrand/tutorials/#sujet44
%This latter one uses the xypackage
%\usepackage{x}

\newcommand\eqnnumber{\addtocounter{equation}{1}\tag{\theequation}} % Allows us to tag the last equation in an align

\usepackage{accents}
\newcommand*{\dt}[1]{%
\accentset{\mbox{\large\bfseries .}}{#1}}
\newcommand*{\ddt}[1]{%
\accentset{\mbox{\large\bfseries .\hspace{-0.25ex}.}}{#1}}
\newcommand{\B}[1]{\mathbf{#1}}
\DeclareMathAlphabet{\V}{OML}{cmm}{b}{it}


% To get decent pdf files:
\usepackage{caption}
\usepackage{subcaption}

\newcommand{\bs}[1]{\boldsymbol{#1}}
\newcommand{\underscore}{\char`\_}
\newcommand{\tildetext}{\char`\~}
\newcommand{\defn}{\stackrel{\triangle}{=}}
\newcommand{\deriv}[2]{\frac{d{#1}}{d{#2}}}
\newcommand{\cas}{\text{cas }}

\lstset{columns=fullflexible,
    xleftmargin=0pt,
    frame=single,
    keepspaces=false,
    tabsize=2,
    breaklines=true,%
    morekeywords={matlab2tikz},
    keywordstyle=\color{blue},%
    morekeywords=[2]{1}, keywordstyle=[2]{\color{black}},
    identifierstyle=\color{black},%
    showstringspaces=false,%without this there will be a symbol in the places where there is a space
    numbers=left,%
    numberstyle={\tiny \color{black}},% size of the numbers
    numbersep=5pt, % this defines how far the numbers are from the text
    emph=[1]{for,end,break},emphstyle=[1]\color{red}, %some words to emphasise
    %emph=[2]{word1,word2}, emphstyle=[2]{style},
}

\newtheorem{theorem}{Theorem}
\newtheorem{definition}{Definition}
\newtheorem{corollary}{Corollary}
\newtheorem{proof}{Proof}

\DeclareMathOperator*{\argmin}{argmin}
\DeclareMathOperator*{\argmax}{argmax}
\DeclareMathOperator*{\sinc}{sinc}
\DeclareMathOperator*{\divergence}{div}

\newcommand{\trunc}[1]{\left[#1\right]_+}
\newcommand{\prob}[1]{\mathbb{P}\left[#1\right]}
\newcommand{\pr}[1]{\text{Pr}\left\{#1\right\}}
\newcommand{\expect}[1]{\mathbb{E}\left[#1\right]}
\newcommand{\llse}[2]{\mathbb{L}\left[#1|#2\right]}
\newcommand{\var}[1]{Var\left(#1\right)}
\newcommand{\cov}[1]{Cov\left(#1\right)}
\newcommand{\normal}[3]{#1\sim\mathcal{N}\left(#2, #3\right)}
\newcommand{\norm}[1]{\|#1\|}
\newcommand{\vecnorm}[1]{\|\vec{#1}\|}
\newcommand{\iid}{independently and identically distributed }
\newcommand{\ip}[2]{\langle #1, #2 \rangle}
\newcommand{\markov}[3]{#1\text{\textemdash}#2\text{\textemdash}#3}
\newcommand{\R}{\mathbb{R}}

\renewcommand{\arraystretch}{1.5}

\DeclareSIUnit\decade{decade}

\tikzstyle{circleblock} = [draw, fill=white, circle, node distance=0.5cm]
\tikzstyle{block} = [draw, fill=white, rectangle, 
    minimum height=3em, minimum width=6em]
\tikzstyle{matrix} = [draw, fill=white, rectangle, 
    minimum height=1em, minimum width=1em]
\tikzstyle{sum} = [draw, fill=white, circle, node distance=1cm]
\tikzstyle{amp} = [regular polygon, regular polygon sides=3, shape border rotate=-90, draw]
\tikzstyle{input} = [coordinate]
\tikzstyle{output} = [coordinate]
\tikzstyle{pinstyle} = [pin edge={to-,thin,black}]
\tikzstyle{int}=[draw, fill=white, minimum size=2em]
\tikzstyle{init} = [pin edge={to-,thin,black}]
\tikzstyle{filter} = [draw, fill=white, rectangle, minimum height=3em, minimum width=6em, align=center, text width=2cm]

\includecomment{gitbook-image}

\begin{document}
\title{EE222 Course Notes}
\author{Anmol Parande}
\date{Spring 2022 - Professors Shankar Shastry and Koushil Srinath}
\maketitle
\textbf{Disclaimer: }These notes reflect EE222 when I took the course (Fall 2021). They may not accurately reflect current course content, so use at your own risk.
If you find any typos, errors, etc, please raise an issue on the \href{https://github.com/parandea17/BerkeleyNotes}{GitHub repository}.
\tableofcontents
\newpage
\section{Nonlinear System Dynamics}
Consider the non-linear system \[
	\diff[]{x}{t} = f(x, t).
\]
\begin{definition}
	The system is autonomous if $f(x, t)$ is not explicitly dependent on time $t$.
	\label{defn:autonomous-system}
\end{definition}
\begin{definition}
	A point $x_0$ is an equilibrium point at time $t_0$ if \[
		\forall t \geq t_0, \ f(x_0, t) = 0 
	\]
	\label{defn:equilibrium-point}
\end{definition}
\subsection{Planar Dynamical Systems}
Planar dynamical systems are those with 2 state variables.
Suppose we linearize the system $\diff[]{\bs{x}}{t} = f(\bs{x})$ at an equilibrium point.
\[
	\diff[]{\bs{x}}{t} = D_f|_{\bs{x} = \bs{x_0}}\bs{x} 
\]
Depending on the eigenvalues of $D_f$, the Jacobian, we get several cases for
how this linear system behaves. We'll let $z_1$ and $z_2$ be the eigenbasis of
the \textit{phase space}.
\begin{enumerate}
	\item The eigenvalues are real, yielding solutions $z_1 = z_1(0)e^{\lambda_1
		t}, z_2 = z_2(0)e^{\lambda_2 t}$. If we eliminate the time variable, we can
		plot the trajectories of the system.
		\[
			\frac{z_1}{z_1(0)} = \left(\frac{z_2}{z_2(0)}\right)^{\frac{\lambda_1}{\lambda_2}}
		\]
		\begin{enumerate}
			\item When $\lambda_1, \lambda_2 < 0$, all trajectories converge to the origin, so we call this a \textbf{stable node}.
			\item When $\lambda_1, \lambda_2 > 0$, all trajectories blow up, so we call this an \textbf{unstable node}.
			\item When $\lambda_1 < 0 < \lambda_2$, the trajectories will converge to
				the origin along the axis corresponding to $\lambda_1$ and diverge along
				the axis corresponding to $\lambda_2$, so we call this a \textbf{saddle node}.
		\end{enumerate}
	\item There is a single repeated eigenvalue with one eigenvector. As before,
		we can eliminate the time variable and plot the trajectories on the $z_1$,
		$z_2$ axes.
		\begin{enumerate}
			\item When $\lambda < 0$, the trajetories will converge to the origin, so
				we call it an \textbf{improper stable node}
			\item When $\lambda > 0$, the trajetories will diverge from the origin, so
				we call it an \textbf{improper unstable node}
		\end{enumerate}
	\item When there is a complex pair of eigenvalues, the linear system will have
		oscillatory behavior. The Real Jordan form of $D_f$ will look like \[
			D_f = \begin{bmatrix} \alpha & \beta \\ -\beta & \alpha \end{bmatrix}.
		\]
		The parameter $\beta$ will determine the direction of the trajectories
		(clockwise if positive).
		\begin{enumerate}
			\item When $\alpha < 0$, the trajectories will spiral towards the origin,
				so we call it a \textbf{stable focus}.
			\item When $\alpha = 0$, the trajectories will remain at a constant radius
				from the origin, so we call it a \textbf{center}.
			\item When $\alpha > 0$, the trajectories will spiral away from the
				origin, so we call it an \textbf{unstable focus}.
		\end{enumerate}
\end{enumerate}
It turns out that understanding the linear dynamics at equilibrium points can be helpful in
understanding the nonlinear dynamics near equilibrium points.
\begin{theorem}[Hartman-Grobman Theorem]
	If the linearization of a planar dynamical system $\diff{\bs{x}}{t} =
	f(\bs{x})$ at an equilibrium point $\bs{x_0}$ has no zero or purely imaginary eigenvalues, then there exists a
	homeomorphism from a neighborhood $U$ of $x_0$ into $\mathbb{R}^2$ which takes
	trajectories of the nonlinear system and maps them onto the linearization
	where $h(\bs{x_0}) = 0$, and the homeomorphism can be chosen to preserve the
	parameterization by time.
	\label{thm:hartman-grobman}
\end{theorem}
\Cref{thm:hartman-grobman} essentially says that the linear dynamics predict the
nonlinear dynamics around equilibria, but only for a neighborhood around the
equilibrium point. Outisde of this neighborhood, the linearization may be very
wrong.

Non-linear systems can also have periodic solutions.
\begin{definition}
	A closed orbit $\gamma$ is a trajectory of the system such that $\gamma(0) =
	\gamma(T)$ for finite $T$.
	\label{defn:closed-orbit}
\end{definition}

Suppose that we have a simply connected region $D$ (meaning $D$ cannot be
contracted to a point) and we want to know if it contains a closed orbit.
\begin{theorem}[Bendixon's Theorem]
	If $\divergence(f)$ is not identically zero in a sub-region of $D$ and does not
	change sign in $D$, then $D$ contains no closed orbits.
	\label{thm:bendixons}
\end{theorem}
\Cref{thm:bendixons} lets us rule out closed orbits from regions of
$\mathbb{R}^2$.
\begin{definition}
	A region $M \subset \mathbb{R}^2$ is positively invariant for a trajectory
	$\phi_t(\bs{x})$ if $\forall x\in M, \forall t \geq 0, \phi_t(\bs{x}) \in M$.
	\label{defn:positive-invariance}
\end{definition}
A positively invariant set essentially means that once a trajectory enters the
set, it cannot leave. That means all of the vector field lines must point inside
the set.
If we have a positively invariant region, then we can determine whether it
contains closed orbits.
\begin{theorem}[Poincare-Bendixson Theorem]
	If $M$ is a compact, positively invariant set for the flow $\phi_t(\bs{x})$,
	then if $M$ contains no equilibrium points, then $M$ has a limit cycle.
	\label{thm:poincare-bendixson}
\end{theorem}
\end{document}

