\documentclass[12pt]{article}

\setlength{\topmargin} {-1in}

% Style modifications
\oddsidemargin  0.25in
\evensidemargin 0.25in
\textwidth      6.0in
%\textheight     8.0in
%\textheight     8.5in % Changed on 13 September 2012
\textheight     9in % Changed on 3 October 2013
\parskip        0.1in
\parindent      0.0in
\headheight     1.0in

\headsep        .25in

\usepackage{graphicx}
\usepackage{amsmath,amsfonts,amssymb,amscd,verbatim,graphicx,fancyhdr,tikz}
\usepackage{mathtools} %for minus plus sign alignment in matrices and vectors, added 13 Feb 2017

\usepackage{pgfplots}
\usetikzlibrary{arrows,automata}
\usetikzlibrary{positioning}
\usetikzlibrary{shapes.geometric}

\usepgfplotslibrary{groupplots}
\pgfplotsset{compat=1.16}

\usepackage{tikz} 
\usetikzlibrary{arrows.meta}
\usepackage{circuitikz}

\usepackage{esdiff}
\usepackage{siunitx}

\usepackage{palatino}
\usepackage{enumerate}
\usepackage{multicol}
\usepackage{listings}
\usepackage{color}
\usepackage{float}
\usepackage{esdiff} % Derivatives
\usepackage{hyperref}
\usepackage{cleveref}
\usepackage{bookmark}
\usepackage{slashbox}
\usepackage{tabularx}
\usepackage{booktabs}
\usepackage{siunitx}

\newcolumntype{Y}{>{\centering\arraybackslash}X}

\hypersetup{
    colorlinks=true,
    linkcolor=blue,
    filecolor=magenta,      
    urlcolor=blue,
}

\newcommand{\bblue}{\color[rgb]{0.2,0.2,0.7}}

\usepackage{mcode}
%Alternative: http://perso.telecom-bretagne.eu/gillesbertrand/tutorials/#sujet44
%This latter one uses the xypackage
%\usepackage{x}

\newcommand\eqnnumber{\addtocounter{equation}{1}\tag{\theequation}} % Allows us to tag the last equation in an align

\usepackage{accents}
\newcommand*{\dt}[1]{%
\accentset{\mbox{\large\bfseries .}}{#1}}
\newcommand*{\ddt}[1]{%
\accentset{\mbox{\large\bfseries .\hspace{-0.25ex}.}}{#1}}
\newcommand{\B}[1]{\mathbf{#1}}
\DeclareMathAlphabet{\V}{OML}{cmm}{b}{it}


% To get decent pdf files:
\usepackage{caption}
\usepackage{subcaption}

\newcommand{\bs}[1]{\boldsymbol{#1}}
\newcommand{\underscore}{\char`\_}
\newcommand{\tildetext}{\char`\~}
\newcommand{\defn}{\stackrel{\triangle}{=}}
\newcommand{\deriv}[2]{\frac{d{#1}}{d{#2}}}
\newcommand{\cas}{\text{cas }}

\lstset{columns=fullflexible,
    xleftmargin=0pt,
    frame=single,
    keepspaces=false,
    tabsize=2,
    breaklines=true,%
    morekeywords={matlab2tikz},
    keywordstyle=\color{blue},%
    morekeywords=[2]{1}, keywordstyle=[2]{\color{black}},
    identifierstyle=\color{black},%
    showstringspaces=false,%without this there will be a symbol in the places where there is a space
    numbers=left,%
    numberstyle={\tiny \color{black}},% size of the numbers
    numbersep=5pt, % this defines how far the numbers are from the text
    emph=[1]{for,end,break},emphstyle=[1]\color{red}, %some words to emphasise
    %emph=[2]{word1,word2}, emphstyle=[2]{style},
}

\newtheorem{theorem}{Theorem}
\newtheorem{definition}{Definition}
\newtheorem{corollary}{Corollary}
\newtheorem{proof}{Proof}

\DeclareMathOperator*{\argmin}{argmin}
\DeclareMathOperator*{\argmax}{argmax}

\newcommand{\prob}[1]{\mathbb{P}\left[#1\right]}
\newcommand{\expect}[1]{\mathbb{E}\left[#1\right]}
\newcommand{\var}[1]{Var\left(#1\right)}
\newcommand{\cov}[1]{Cov\left(#1\right)}
\newcommand{\normal}[3]{#1\sim\mathcal{N}\left(#2, #3\right)}
\newcommand{\norm}[1]{\|#1\|}
\newcommand{\vecnorm}[1]{\|\vec{#1}\|}
\newcommand{\iid}{independently and identically distributed }
\renewcommand{\arraystretch}{1.5}

\DeclareSIUnit\decade{decade}

\tikzstyle{circleblock} = [draw, fill=white, circle, node distance=0.5cm]
\tikzstyle{block} = [draw, fill=white, rectangle, 
    minimum height=3em, minimum width=6em]
\tikzstyle{matrix} = [draw, fill=white, rectangle, 
    minimum height=1em, minimum width=1em]
\tikzstyle{sum} = [draw, fill=white, circle, node distance=1cm]
\tikzstyle{amp} = [regular polygon, regular polygon sides=3, shape border rotate=-90, draw]
\tikzstyle{input} = [coordinate]
\tikzstyle{output} = [coordinate]
\tikzstyle{pinstyle} = [pin edge={to-,thin,black}]
\tikzstyle{int}=[draw, fill=blue!20, minimum size=2em]
\tikzstyle{init} = [pin edge={to-,thin,black}]


\begin{document}
\title{EECS126 Course Notes}
\author{Anmol Parande}
\date{Spring 2021 - Professor Thomas Courtade}
\maketitle
\textbf{Disclaimer: }These notes reflect 126 when I took the course (Spring 2021). They may not accurately reflect current course content, so use at your own risk.
If you find any typos, errors, etc, please raise an issue on the \href{https://github.com/parandea17/BerkeleyNotes}{GitHub repository}.
\tableofcontents
\newpage
\section{Introduction to Probability}
\begin{definition}
  A probability space is a triple $(\Omega, \mathcal{F}, P)$ where $\Omega$ is a set of objects called the sample space, $\mathcal{F}$ is a family of subsets of $\Omega$ called events, and the probability measure $P:\mathcal{F}\rightarrow [0,1]$.
  \label{defn:prob-space}
\end{definition}
One key assumption we make is that $\mathcal{F}$ is a $\sigma$-algebra containing $\Omega$, meaning that countably many complements, unions, and intersections of events in $\mathcal{F}$ are also events in $\mathcal{F}$.
The probability measure $P$ must obey \textbf{Kolmogorov's Axioms}.
\begin{enumerate}
  \item $\forall A \in \mathcal{F},\ P(A) \geq 0$
  \item $P(\Omega) = 1$
  \item If $A_1, A_2, \cdots\in \mathcal{F}$ and $\forall i\ne j,\ A_i\bigcap A_j=\emptyset$, then $P\left(\bigcup_{i\geq 1}A_i\right) = \sum_{i\geq1}P(A_i)$
\end{enumerate}
We choose $\Omega$ and $\mathcal{F}$ to model problems in a way that makes our calculations easy.
\begin{theorem}
  \[
	P(A^c) = 1 - P(A)
  \]
  \label{thm:complementarity}
\end{theorem}
\begin{theorem}[Inclusion-Exclusion Principle]
  \[
	P\left( \bigcup_{i=1}^{n}A_i \right) = \sum_{k=1}^{n}(-1)^{k+1}\left( \sum_{1\leq i_1<\cdots<i_k\leq n} P(A_{i_1}\cap \cdots \cap A_{i_k}) \right)
  \]
  \label{thm:inclusion-exclusion}
\end{theorem}
\begin{theorem}[Law of Total Probability]
  If $A_1, A_2, \cdots$ partition $\Omega$ (i.e $A_i$ are disjoint and $\cup A_i = \Omega$), then for event $B$,
  \[
	P(B) = \sum_iP(B\cap A_i)
  \]
  \label{thm:total-prob}
\end{theorem}
\subsection{Conditional Probability}
\begin{definition}
  If $B$ is an event with $P(B)>0$, then the conditional probability of $A$ given $B$ is
  \[
	P(A|B) = \frac{P(A\cap B)}{P(B)}
  \]
  \label{defn:conditional-prob}
\end{definition}
Intuitively, conditional probabilty is the probability of event $A$ given that event $B$ has occurred.
In terms of probability spaces, it is as if we have taken $(\Omega, \mathcal{F}, P)$ and now have a probabilty measure $P(\cdot|C)$ belonging to the space $(\Omega, \mathcal{F}, P(\cdot|C))$.
\begin{theorem}[Bayes Theorem]
  \[
	P(A|B) = \frac{P(B|A)P(A)}{P(B)}
  \]
  \label{thm:bayes}
\end{theorem}
\subsection{Independence}
\begin{definition}
  Events $A$ and $B$ are independent if $P(A\cap B) = P(A)P(B)$
  \label{defn:independence}
\end{definition}
If $P(B)>0$, then $A, B$ are independent if and only if $P(A|B) = P(A)$. In other words, knowing $B$ occurred gave no extra information about $A$.
\begin{definition}
  If $A,B,C$ with $P(C)>0$ satisfy $P(A\cap B|C) = P(A|C)P(B|C)$, then $A$ and $B$ are conditionally independent given $C$.
  \label{defn:conditional-independence}
\end{definition}
Conditional independence is a special case of independence where $A$ and $B$ are not necessarily independent in the original probability space which has the measure $P$, but are independent in the new probability space conditioned on $C$ with the measure $P(\cdot|C)$.
\section{Discrete Probability}
\begin{definition}
  A random variable is a function $X:\Omega\rightarrow\mathbb{R}$ with the property $\forall \alpha\in\mathbb{R},\ \{\omega\in\Omega:\ X(\omega) \leq \alpha\} \in \mathcal{F}$.
  \label{defn:drv}
\end{definition}
The condition in \cref{defn:drv} is necessary to compute $P(X\leq \alpha),\ \forall \alpha\in\mathbb{R}$.
In a sense, it binds the probability space to the random variable.
\end{document}

